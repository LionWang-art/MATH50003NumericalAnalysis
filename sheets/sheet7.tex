\documentclass[12pt,a4paper]{article}

\usepackage[a4paper,text={16.5cm,25.2cm},centering]{geometry}
\usepackage{lmodern}
\usepackage{amssymb,amsmath}
\usepackage{bm}
\usepackage{graphicx}
\usepackage{microtype}
\usepackage{hyperref}
\usepackage[usenames,dvipsnames]{xcolor}
\setlength{\parindent}{0pt}
\setlength{\parskip}{1.2ex}




\hypersetup
       {   pdfauthor = {  },
           pdftitle={  },
           colorlinks=TRUE,
           linkcolor=black,
           citecolor=blue,
           urlcolor=blue
       }




\usepackage{upquote}
\usepackage{listings}
\usepackage{xcolor}
\lstset{
    basicstyle=\ttfamily\footnotesize,
    upquote=true,
    breaklines=true,
    breakindent=0pt,
    keepspaces=true,
    showspaces=false,
    columns=fullflexible,
    showtabs=false,
    showstringspaces=false,
    escapeinside={(*@}{@*)},
    extendedchars=true,
}
\newcommand{\HLJLt}[1]{#1}
\newcommand{\HLJLw}[1]{#1}
\newcommand{\HLJLe}[1]{#1}
\newcommand{\HLJLeB}[1]{#1}
\newcommand{\HLJLo}[1]{#1}
\newcommand{\HLJLk}[1]{\textcolor[RGB]{148,91,176}{\textbf{#1}}}
\newcommand{\HLJLkc}[1]{\textcolor[RGB]{59,151,46}{\textit{#1}}}
\newcommand{\HLJLkd}[1]{\textcolor[RGB]{214,102,97}{\textit{#1}}}
\newcommand{\HLJLkn}[1]{\textcolor[RGB]{148,91,176}{\textbf{#1}}}
\newcommand{\HLJLkp}[1]{\textcolor[RGB]{148,91,176}{\textbf{#1}}}
\newcommand{\HLJLkr}[1]{\textcolor[RGB]{148,91,176}{\textbf{#1}}}
\newcommand{\HLJLkt}[1]{\textcolor[RGB]{148,91,176}{\textbf{#1}}}
\newcommand{\HLJLn}[1]{#1}
\newcommand{\HLJLna}[1]{#1}
\newcommand{\HLJLnb}[1]{#1}
\newcommand{\HLJLnbp}[1]{#1}
\newcommand{\HLJLnc}[1]{#1}
\newcommand{\HLJLncB}[1]{#1}
\newcommand{\HLJLnd}[1]{\textcolor[RGB]{214,102,97}{#1}}
\newcommand{\HLJLne}[1]{#1}
\newcommand{\HLJLneB}[1]{#1}
\newcommand{\HLJLnf}[1]{\textcolor[RGB]{66,102,213}{#1}}
\newcommand{\HLJLnfm}[1]{\textcolor[RGB]{66,102,213}{#1}}
\newcommand{\HLJLnp}[1]{#1}
\newcommand{\HLJLnl}[1]{#1}
\newcommand{\HLJLnn}[1]{#1}
\newcommand{\HLJLno}[1]{#1}
\newcommand{\HLJLnt}[1]{#1}
\newcommand{\HLJLnv}[1]{#1}
\newcommand{\HLJLnvc}[1]{#1}
\newcommand{\HLJLnvg}[1]{#1}
\newcommand{\HLJLnvi}[1]{#1}
\newcommand{\HLJLnvm}[1]{#1}
\newcommand{\HLJLl}[1]{#1}
\newcommand{\HLJLld}[1]{\textcolor[RGB]{148,91,176}{\textit{#1}}}
\newcommand{\HLJLs}[1]{\textcolor[RGB]{201,61,57}{#1}}
\newcommand{\HLJLsa}[1]{\textcolor[RGB]{201,61,57}{#1}}
\newcommand{\HLJLsb}[1]{\textcolor[RGB]{201,61,57}{#1}}
\newcommand{\HLJLsc}[1]{\textcolor[RGB]{201,61,57}{#1}}
\newcommand{\HLJLsd}[1]{\textcolor[RGB]{201,61,57}{#1}}
\newcommand{\HLJLsdB}[1]{\textcolor[RGB]{201,61,57}{#1}}
\newcommand{\HLJLsdC}[1]{\textcolor[RGB]{201,61,57}{#1}}
\newcommand{\HLJLse}[1]{\textcolor[RGB]{59,151,46}{#1}}
\newcommand{\HLJLsh}[1]{\textcolor[RGB]{201,61,57}{#1}}
\newcommand{\HLJLsi}[1]{#1}
\newcommand{\HLJLso}[1]{\textcolor[RGB]{201,61,57}{#1}}
\newcommand{\HLJLsr}[1]{\textcolor[RGB]{201,61,57}{#1}}
\newcommand{\HLJLss}[1]{\textcolor[RGB]{201,61,57}{#1}}
\newcommand{\HLJLssB}[1]{\textcolor[RGB]{201,61,57}{#1}}
\newcommand{\HLJLnB}[1]{\textcolor[RGB]{59,151,46}{#1}}
\newcommand{\HLJLnbB}[1]{\textcolor[RGB]{59,151,46}{#1}}
\newcommand{\HLJLnfB}[1]{\textcolor[RGB]{59,151,46}{#1}}
\newcommand{\HLJLnh}[1]{\textcolor[RGB]{59,151,46}{#1}}
\newcommand{\HLJLni}[1]{\textcolor[RGB]{59,151,46}{#1}}
\newcommand{\HLJLnil}[1]{\textcolor[RGB]{59,151,46}{#1}}
\newcommand{\HLJLnoB}[1]{\textcolor[RGB]{59,151,46}{#1}}
\newcommand{\HLJLoB}[1]{\textcolor[RGB]{102,102,102}{\textbf{#1}}}
\newcommand{\HLJLow}[1]{\textcolor[RGB]{102,102,102}{\textbf{#1}}}
\newcommand{\HLJLp}[1]{#1}
\newcommand{\HLJLc}[1]{\textcolor[RGB]{153,153,119}{\textit{#1}}}
\newcommand{\HLJLch}[1]{\textcolor[RGB]{153,153,119}{\textit{#1}}}
\newcommand{\HLJLcm}[1]{\textcolor[RGB]{153,153,119}{\textit{#1}}}
\newcommand{\HLJLcp}[1]{\textcolor[RGB]{153,153,119}{\textit{#1}}}
\newcommand{\HLJLcpB}[1]{\textcolor[RGB]{153,153,119}{\textit{#1}}}
\newcommand{\HLJLcs}[1]{\textcolor[RGB]{153,153,119}{\textit{#1}}}
\newcommand{\HLJLcsB}[1]{\textcolor[RGB]{153,153,119}{\textit{#1}}}
\newcommand{\HLJLg}[1]{#1}
\newcommand{\HLJLgd}[1]{#1}
\newcommand{\HLJLge}[1]{#1}
\newcommand{\HLJLgeB}[1]{#1}
\newcommand{\HLJLgh}[1]{#1}
\newcommand{\HLJLgi}[1]{#1}
\newcommand{\HLJLgo}[1]{#1}
\newcommand{\HLJLgp}[1]{#1}
\newcommand{\HLJLgs}[1]{#1}
\newcommand{\HLJLgsB}[1]{#1}
\newcommand{\HLJLgt}[1]{#1}


\def\endash{–}
\def\bbD{ {\mathbb D} }
\def\bbZ{ {\mathbb Z} }
\def\bbR{ {\mathbb R} }
\def\bbC{ {\mathbb C} }

\def\x{ {\vc x} }
\def\a{ {\vc a} }
\def\b{ {\vc b} }
\def\e{ {\vc e} }
\def\f{ {\vc f} }
\def\u{ {\vc u} }
\def\v{ {\vc v} }
\def\y{ {\vc y} }
\def\z{ {\vc z} }

\def\red#1{ {\color{red} #1} }
\def\blue#1{ {\color{blue} #1} }
\def\green#1{ {\color{ForestGreen} #1} }
\def\magenta#1{ {\color{magenta} #1} }

\input{somacros}

\begin{document}



\textbf{Numerical Analysis MATH50003 (2023\ensuremath{\endash}24) Problem Sheet 7}

\textbf{Problem 1(a)} Show for a unitary matrix $Q \ensuremath{\in} U(n)$ and a vector $\ensuremath{\bm{\x}} \ensuremath{\in} \ensuremath{\bbC}^n$ that multiplication by $Q$ preserve the 2-norm: $\|Q \ensuremath{\bm{\x}}\| = \|\ensuremath{\bm{\x}}\|.$

\textbf{Problem 1(b)} Show that the eigenvalues $\ensuremath{\lambda}$ of a unitary matrix $Q$ are on the unit circle: $|\ensuremath{\lambda}| = 1$. Hint: recall for any eigenvalue $\ensuremath{\lambda}$ that there exists a unit eigenvector $\ensuremath{\bm{\v}} \ensuremath{\in} \ensuremath{\bbC}^n$ (satisfying $\| \ensuremath{\bm{\v}} \| = 1$). 

\textbf{Problem 1(c)} Show for an orthogonal matrix $Q \ensuremath{\in} O(n)$ that $\det Q = \ensuremath{\pm}1$. Give an example of $Q \ensuremath{\in} U(n)$ such that $\det Q \ensuremath{\neq} \ensuremath{\pm}1$. Hint: recall for any real matrices $A$ and $B$ that $\det A = \det A^\ensuremath{\top}$ and $\det(AB) = \det A \det B$.

\textbf{Problem 1(d)} A normal matrix commutes with its adjoint. Show that $Q \ensuremath{\in} U(n)$ is normal.

\textbf{Problem 1(e)}  The spectral theorem states that any normal matrix is unitarily diagonalisable: if $A$ is normal then $A = V \ensuremath{\Lambda} V^\ensuremath{\star}$ where $V \ensuremath{\in} U(n)$ and $\ensuremath{\Lambda}$ is diagonal. Use this to show that  $Q \ensuremath{\in} U(n)$ is equal to $I$ if and only if all its eigenvalues are 1.

\textbf{Problem 2} Consider the vectors
\[
\ensuremath{\bm{\a}} = \begin{bmatrix} 1 \\ 2 \\ 2 \end{bmatrix}\qquad\hbox{and}\qquad  \ensuremath{\bm{\b}} = \begin{bmatrix} 1 \\ 2{\rm i} \\ 2 \end{bmatrix}.
\]
Use reflections to determine the entries of orthogonal/unitary matrices $Q_1, Q_2, Q_3$ such that
\[
Q_1 \ensuremath{\bm{\a}} = \begin{bmatrix} 3 \\ 0 \\ 0 \end{bmatrix}, Q_2 \ensuremath{\bm{\a}} = \begin{bmatrix} -3 \\ 0 \\ 0 \end{bmatrix},
Q_3 \ensuremath{\bm{\b}} = \begin{bmatrix} -3 \\ 0 \\ 0 \end{bmatrix}
\]
\textbf{Problem 3(a)} What simple rotation matrices $Q_1,Q_2 \ensuremath{\in} SO(2)$ have the property that:
\[
Q_1 \begin{bmatrix} 1 \\ 2 \end{bmatrix} =\begin{bmatrix} \sqrt{5} \\ 0 \end{bmatrix},  Q_2 \begin{bmatrix} \sqrt{5} \\ 2 \end{bmatrix} =  \begin{bmatrix} 3 \\ 0 \end{bmatrix}
\]
\textbf{Problem 3(b)} Find an orthogonal matrix that is a product of two simple rotations but acting on two different subspaces:
\[
Q  = \underbrace{\begin{bmatrix} \cos \ensuremath{\theta}_2 & & -\sin \ensuremath{\theta}_2  \\ & 1 \\
\sin \ensuremath{\theta}_2  & & \cos \ensuremath{\theta}_2  \end{bmatrix}}_{Q_2} \underbrace{\begin{bmatrix} \cos \ensuremath{\theta}_1 & -\sin \ensuremath{\theta}_1  \\ \sin \ensuremath{\theta}_1 & \cos \ensuremath{\theta}_1 \\ && 1 \end{bmatrix}}_{Q_1}
\]
so that, for $\ensuremath{\bm{\a}}$ defined above,
\[
Q \ensuremath{\bm{\a}}  = \begin{bmatrix} \|\ensuremath{\bm{\a}}\| \\ 0 \\ 0 \end{bmatrix}.
\]
Hint: you do not need to determine $\ensuremath{\theta}_1, \ensuremath{\theta}_2$, instead you can write the entries of  $Q_1, Q_2$  directly using just square-roots. 

\textbf{Problem 4(a)} Show that every matrix $A \ensuremath{\in} \ensuremath{\bbR}^{m \ensuremath{\times} n}$ has a  QR factorisation such that the diagonal of $R$ is non-negative. Make sure to include the case of more columns than rows (i.e. $m < n$). 

\textbf{Problem 4(b)} Show that the QR factorisation of a square invertible matrix $A \ensuremath{\in} \ensuremath{\bbR}^{n \ensuremath{\times} n}$ is unique, provided that the diagonal of $R$ is positive.



\end{document}