\documentclass[12pt,a4paper]{book}

\usepackage[a4paper,text={16.5cm,25.2cm},centering]{geometry}
\usepackage{lmodern}
\usepackage{amssymb,amsmath}
\usepackage{bm}
\usepackage{graphicx}
\usepackage{microtype}
\usepackage{hyperref}
\usepackage{amsthm}
\usepackage{upquote}
\usepackage{listings}
\usepackage{appendix}
\usepackage[usenames,dvipsnames]{xcolor}
\setlength{\parindent}{0pt}
\setlength{\parskip}{1.2ex}

\lstset{
    basicstyle=\ttfamily\footnotesize,
    upquote=true,
    breaklines=true,
    breakindent=0pt,
    keepspaces=true,
    showspaces=false,
    columns=fullflexible,
    showtabs=false,
    showstringspaces=false,
    escapeinside={(*@}{@*)},
    extendedchars=true,
}

\newcommand{\HLJLt}[1]{#1}
\newcommand{\HLJLw}[1]{#1}
\newcommand{\HLJLe}[1]{#1}
\newcommand{\HLJLeB}[1]{#1}
\newcommand{\HLJLo}[1]{#1}
\newcommand{\HLJLk}[1]{\textcolor[RGB]{148,91,176}{\textbf{#1}}}
\newcommand{\HLJLkc}[1]{\textcolor[RGB]{59,151,46}{\textit{#1}}}
\newcommand{\HLJLkd}[1]{\textcolor[RGB]{214,102,97}{\textit{#1}}}
\newcommand{\HLJLkn}[1]{\textcolor[RGB]{148,91,176}{\textbf{#1}}}
\newcommand{\HLJLkp}[1]{\textcolor[RGB]{148,91,176}{\textbf{#1}}}
\newcommand{\HLJLkr}[1]{\textcolor[RGB]{148,91,176}{\textbf{#1}}}
\newcommand{\HLJLkt}[1]{\textcolor[RGB]{148,91,176}{\textbf{#1}}}
\newcommand{\HLJLn}[1]{#1}
\newcommand{\HLJLna}[1]{#1}
\newcommand{\HLJLnb}[1]{#1}
\newcommand{\HLJLnbp}[1]{#1}
\newcommand{\HLJLnc}[1]{#1}
\newcommand{\HLJLncB}[1]{#1}
\newcommand{\HLJLnd}[1]{\textcolor[RGB]{214,102,97}{#1}}
\newcommand{\HLJLne}[1]{#1}
\newcommand{\HLJLneB}[1]{#1}
\newcommand{\HLJLnf}[1]{\textcolor[RGB]{66,102,213}{#1}}
\newcommand{\HLJLnfm}[1]{\textcolor[RGB]{66,102,213}{#1}}
\newcommand{\HLJLnp}[1]{#1}
\newcommand{\HLJLnl}[1]{#1}
\newcommand{\HLJLnn}[1]{#1}
\newcommand{\HLJLno}[1]{#1}
\newcommand{\HLJLnt}[1]{#1}
\newcommand{\HLJLnv}[1]{#1}
\newcommand{\HLJLnvc}[1]{#1}
\newcommand{\HLJLnvg}[1]{#1}
\newcommand{\HLJLnvi}[1]{#1}
\newcommand{\HLJLnvm}[1]{#1}
\newcommand{\HLJLl}[1]{#1}
\newcommand{\HLJLld}[1]{\textcolor[RGB]{148,91,176}{\textit{#1}}}
\newcommand{\HLJLs}[1]{\textcolor[RGB]{201,61,57}{#1}}
\newcommand{\HLJLsa}[1]{\textcolor[RGB]{201,61,57}{#1}}
\newcommand{\HLJLsb}[1]{\textcolor[RGB]{201,61,57}{#1}}
\newcommand{\HLJLsc}[1]{\textcolor[RGB]{201,61,57}{#1}}
\newcommand{\HLJLsd}[1]{\textcolor[RGB]{201,61,57}{#1}}
\newcommand{\HLJLsdB}[1]{\textcolor[RGB]{201,61,57}{#1}}
\newcommand{\HLJLsdC}[1]{\textcolor[RGB]{201,61,57}{#1}}
\newcommand{\HLJLse}[1]{\textcolor[RGB]{59,151,46}{#1}}
\newcommand{\HLJLsh}[1]{\textcolor[RGB]{201,61,57}{#1}}
\newcommand{\HLJLsi}[1]{#1}
\newcommand{\HLJLso}[1]{\textcolor[RGB]{201,61,57}{#1}}
\newcommand{\HLJLsr}[1]{\textcolor[RGB]{201,61,57}{#1}}
\newcommand{\HLJLss}[1]{\textcolor[RGB]{201,61,57}{#1}}
\newcommand{\HLJLssB}[1]{\textcolor[RGB]{201,61,57}{#1}}
\newcommand{\HLJLnB}[1]{\textcolor[RGB]{59,151,46}{#1}}
\newcommand{\HLJLnbB}[1]{\textcolor[RGB]{59,151,46}{#1}}
\newcommand{\HLJLnfB}[1]{\textcolor[RGB]{59,151,46}{#1}}
\newcommand{\HLJLnh}[1]{\textcolor[RGB]{59,151,46}{#1}}
\newcommand{\HLJLni}[1]{\textcolor[RGB]{59,151,46}{#1}}
\newcommand{\HLJLnil}[1]{\textcolor[RGB]{59,151,46}{#1}}
\newcommand{\HLJLnoB}[1]{\textcolor[RGB]{59,151,46}{#1}}
\newcommand{\HLJLoB}[1]{\textcolor[RGB]{102,102,102}{\textbf{#1}}}
\newcommand{\HLJLow}[1]{\textcolor[RGB]{102,102,102}{\textbf{#1}}}
\newcommand{\HLJLp}[1]{#1}
\newcommand{\HLJLc}[1]{\textcolor[RGB]{153,153,119}{\textit{#1}}}
\newcommand{\HLJLch}[1]{\textcolor[RGB]{153,153,119}{\textit{#1}}}
\newcommand{\HLJLcm}[1]{\textcolor[RGB]{153,153,119}{\textit{#1}}}
\newcommand{\HLJLcp}[1]{\textcolor[RGB]{153,153,119}{\textit{#1}}}
\newcommand{\HLJLcpB}[1]{\textcolor[RGB]{153,153,119}{\textit{#1}}}
\newcommand{\HLJLcs}[1]{\textcolor[RGB]{153,153,119}{\textit{#1}}}
\newcommand{\HLJLcsB}[1]{\textcolor[RGB]{153,153,119}{\textit{#1}}}
\newcommand{\HLJLg}[1]{#1}
\newcommand{\HLJLgd}[1]{#1}
\newcommand{\HLJLge}[1]{#1}
\newcommand{\HLJLgeB}[1]{#1}
\newcommand{\HLJLgh}[1]{#1}
\newcommand{\HLJLgi}[1]{#1}
\newcommand{\HLJLgo}[1]{#1}
\newcommand{\HLJLgp}[1]{#1}
\newcommand{\HLJLgs}[1]{#1}
\newcommand{\HLJLgsB}[1]{#1}
\newcommand{\HLJLgt}[1]{#1}


\let\QED=\blacksquare
\def\bbD{{\mathbb D}}
\def\bbZ{{\mathbb Z}}
\def\bbN{{\mathbb N}}
\def\bbF{{\mathbb F}}
\def\bbR{{\mathbb R}}
\def\bbT{{\mathbb T}}
\def\bbC{{\mathbb C}}
\def\emdash{\hbox{---}}
\def\endash{\hbox{--}}
\def\nsubset{\not\subset}
\def\ldq{``}
\def\x{{\vc x}}
\def\a{{\vc a}}
\def\b{{\vc b}}
\def\q{{\vc q}}
\def\c{{\vc c}}
\def\e{{\vc e}}
\def\f{{\vc f}}
\def\u{{\vc u}}
\def\w{{\vc w}}
\def\v{{\vc v}}
\def\y{{\vc y}}
\def\z{{\vc z}}
\def\k{{\vc k}}
\def\vchatf{{\vc {\hat f}}}
\def\zero{{\vc 0}}
\def\Lt{{\tilde L}}
\def\Pt{{\tilde P}}
\def\pt{{\tilde p}}
\def\Ut{{\tilde U}}
\def\baralpha{\bar\alpha}
\def\At{\tilde A}
\def\Rt{\tilde R}
\def\red#1{{\color{red} #1}}
\def\blue#1{{\color{blue} #1}}
\def\green#1{{\color{ForestGreen} #1}}
\def\euler{\E}
\def\ocaret{\wedge\mkern-19mu \bigcirc\,}

\def\fldown{{\rm fl}^{\rm down}}
\def\flup{{\rm fl}^{\rm up}}

\hypersetup
       {   pdfauthor = { {{Sheehan Olver}} },
           pdftitle={ {{MATH50003 Numerical Analysis}} },
           colorlinks=TRUE,
           linkcolor=black,
           citecolor=blue,
           urlcolor=blue
       }

\title{ MATH50003 Numerical Analysis }


\newtheorem{lemma}{Lemma}
\newtheorem{theorem}{Theorem}
\newtheorem{proposition}{Proposition}
\newtheorem{corollary}{Corollary}

\theoremstyle{definition}
\newtheorem{definition}{Definition}
\newtheorem{example}{Example}

\author{ Sheehan Olver }
\renewcommand{\thechapter}{\Roman{chapter}}

\input{somacros}

\begin{document}

\maketitle

\tableofcontents

\chapter{Calculus on a Computer}

In this first chapter we explore the basics of mathematical computing and numerical analysis.
In particular we investigate the following mathematical problems which can not in general be solved exactly:

\begin{enumerate}
\item Integration. General integrals have no closed form expressions. Can we use a computer to approximate the values of definite integrals?
\item Differentiation. Differentiating a formula as in calculus is usually algorithmic, however, it is often needed to compute derivatives without access to an underlying formula, eg,  a function defined only in code. Can we use a computer to approximate derivatives?  A very important application is in Machine Learning, where there is a need to compute gradients to determine the ``right" weights in a neural network. 
\item Root finding. There is no general formula for finding roots (zeros) of arbitrary functions, or even polynomials that are of degree 5 (quintics) or higher. Can we compute roots of general functions using a computer?
\end{enumerate}

In this chapter we discuss:

\begin{enumerate}
\item I.1 Rectangular rule: we review the rectangular rule for integration and deduce the {\it converge rate} of the approximation. In the lab/problem sheet  we investigate its implementation as well as extensions to the Trapezium rule. 
\item I.2 Divided differences: we investigate approximating derivatives by a divided difference and again deduce the convergence rates. In the lab/problem sheet we extend the approach to the central differences formula and computing second derivatives. We also observe a mystery: the approximations may have significant errors in practice, and there is a limit to the accuracy.
\item I.3 Dual numbers: we introduce the algebraic notion of a {\it dual number} which allows the implemention of {\it forward-mode automatic differentiation}, a high accuracy alternative to divided differences for computing derivatives.
\item I.4 Newton's method: Newton's method is a basic approach for computing roots/zeros of a function. We use dual numbers to implement this algorithm.
\end{enumerate}



\input{I.1.RectangularRule.tex}

\section{Divided Differences}
Given a function, how can we approximate its derivative at a point? We consider an intuitive approach to this problem using \emph{(Right-sided) Divided Differences}: 
\[
f'(x) \ensuremath{\approx} {f(x+h) - f(x) \over h}
\]
Note by the definition of the derivative we know that this approximation will converge to the true derivative as $h \ensuremath{\rightarrow} 0$. But in numerical approimxations we also need to consider the rate of convergence. 

Now in the previous section I mentioned there are three basic tools in analysis:  (1) integration-by-parts, (2) geometric series or (3) Taylor series. In this case we use (3):

\begin{proposition}[divided differences error] Suppose that $f$ is twice-differentiable on the interval $[x,x+h]$. The error in approximating the derivative using divided differences is
\[
f'(x) = {f(x+h) - f(x) \over h} + \ensuremath{\delta}
\]
where $|\ensuremath{\delta}| \ensuremath{\leq} Mh/2$ for  $M = \sup_{x \ensuremath{\leq} t \ensuremath{\leq} x+h} |f''(t)|$.

\end{proposition}
\textbf{Proof} Follows immediately from Taylor's theorem:
\[
f(x+h) = f(x) + f'(x) h + \underbrace{{f''(t) \over 2} h^2}_{h \ensuremath{\delta}}
\]
for some $x \ensuremath{\leq} t \ensuremath{\leq} x+h$, by bounding:
\[
|\ensuremath{\delta}| \ensuremath{\leq} \abs{{f''(t) \over 2} h} \ensuremath{\leq} {M  h \over 2}.
\]
\ensuremath{\QED}

Unlike the rectangular rule, the computational cost of computing the divided difference is independent of $h$! We only need to evaluate a function $f$ twice and do a single division. Here we are assuming that the computational cost of evaluating $f$ is independent of the point of evaluation. Later we will investigate the details of how computers work with numbers via floating point,  and confirm that this is a sensible assumption.

So why not just set $h$ ridiculously small? In the lab we explore this question and observe that there are significant errors introduced in the numerical realisation of this algorithm. We will return to the question of understanding these errors after learning floating point numbers. 

There are alternative versions of divided differences. Left-side divided differences evaluates to the left of the point where we wish to know the derivative:
\[
f'(x) \ensuremath{\approx} {f(x) - f(x-h) \over h}
\]
and central differences:
\[
f'(x) \ensuremath{\approx} {f(x + h) - f(x - h) \over 2h}
\]
We can further arrive at an approximation to the second derivative by composing a left- and right-sided finite difference:
\[
f''(x) \ensuremath{\approx} {f'(x+h) - f'(x) \over h} \ensuremath{\approx} {{f(x+h) - f(x) \over h} - {f(x) - f(x-h) \over h} \over h}
= {f(x+h) - 2f(x)  + f(x-h) \over h^2}
\]
In the lab we investigate the convergence rate of these approximations (in particular, that  central differences is more accurate than standard divided differences) and observe that they too suffer from unexplained (for now) loss of accuracy as $h \ensuremath{\rightarrow} 0$. In the problem sheet we prove the theoretical convergence rate, which is never realised because of these errors.




\input{I.3.DualNumbers.tex}
\input{I.4.NewtonMethod.tex}


\chapter{Representing Numbers}

In this chapter we aim to answer the question: when can we rely on computations done on a computer?  Why are some computations (differentiation via divided differences), extremely inaccurate whilst others (integration via rectangular rule) accurate up to about 16 digits?  In order to address these questions we need to dig deeper and understand at a basic level what a computer is actually doing when manipulating numbers. 

Before we begin it is important to have a basic model of how a computer works. Our simplified model of a computer will consist of a \href{https://en.wikipedia.org/wiki/Central_processing_unit}{Central Processing Unit (CPU)}\ensuremath{\emdash}the  brains of the computer\ensuremath{\emdash}and \href{https://en.wikipedia.org/wiki/Computer_data_storage#Primary_storage}{Memory}\ensuremath{\emdash}where  data is stored. Inside the CPU there are \href{https://en.wikipedia.org/wiki/Processor_register}{registers}, where data is temporarily stored after being loaded from memory, manipulated by the CPU, then stored back to memory.  Memory is a sequence of bits: \texttt{1}s and \texttt{0}s, essentially ``on/off" switches, and memory is {\it finite}.  Finally, if one has a $p$-bit CPU (eg a 32-bit or 64-bit CPU), each register consists of exactly $p$-bits. Most likely $p = 64$ on your machine. 


Thus representing numbers on a computer must overcome three fundamental limitations:
\begin{enumerate}
\item CPUs can only manipulate data $p$-bits at a time.
\item Memory is finite (in particular at most $2^p$ bytes).
\item There is no such thing as an ``error'': if anything goes wrong in the computation we must use some of the $p$-bits to indicate this.
\end{enumerate}

This is clearly problematic: there are an infinite number of integers and an uncountable number of reals! Each of which we need to store in precisely $p$-bits. Moreover, some operations are simply undefined, like division by 0.  This chapter discusses the solution used to this problem, alongside the mathematical analysis that is needed to understand the implications, in particular, that computations have {\it error}.

In particular we discuss:

\begin{enumerate}
\item II.1 Integers: unsigned (non-negative) and signed integers are representable using exactly $p$-bits by using modular arithmetic in all operations.
\item II.2 Reals:  real numbers are approximated by floating point numbers, which are a computers version of scientific notation.
\item II.3 Floating Point Arithmetic:  arithmetic with floating point numbers is exact up-to-rounding, which introduces small-but-understandable errors in the computations. We explain how these errors can be analysed mathematically to get rigorous bounds. 
\item II.4 Interval Arithmetic: rounding can be controlled in order to implement {\it interval arithmetic}, a way to compute rigorous bounds for computations. In the lab, we use this to compute up to 15 digits of ${\rm e} \equiv \exp 1$ rigorously with precise bounds on the error.
\end{enumerate}


\input{II.1.Integers}
\input{II.2.Reals}

\section{Floating Point Arithmetic}
Arithmetic operations on floating-point numbers are  \emph{exact up to rounding}. There are three basic rounding strategies: round up/down/nearest. Mathematically we introduce a function to capture the notion of rounding:

\begin{definition}[rounding] ${\rm fl}^{\rm up}_{\ensuremath{\sigma},Q,S} : \mathbb R \rightarrow F_{\ensuremath{\sigma},Q,S}$ denotes the function that rounds a real number up to the nearest floating-point number that is greater or equal. ${\rm fl}^{\rm down}_{\ensuremath{\sigma},Q,S} : \mathbb R \rightarrow F_{\ensuremath{\sigma},Q,S}$ denotes the function that rounds a real number down to the nearest floating-point number that is greater or equal. ${\rm fl}^{\rm nearest}_{\ensuremath{\sigma},Q,S} : \mathbb R \rightarrow F_{\ensuremath{\sigma},Q,S}$ denotes the function that rounds a real number to the nearest floating-point number. In case of a tie, it returns the floating-point number whose least significant bit is equal to zero. We use the notation ${\rm fl}$ when $\ensuremath{\sigma},Q,S$ and the rounding mode are implied by context, with ${\rm fl}^{\rm nearest}$ being the default rounding mode. \end{definition}

In more detail on the behaviour of nearest mode, if a positive number $x$ is between two normal floats $x_- \ensuremath{\leq} x \ensuremath{\leq} x_+$ we can write its expansion as
\[
x = 2^{\green{q}-\ensuremath{\sigma}} (1.\blue{b_1b_2\ensuremath{\ldots}b_S}\red{b_{S+1}\ensuremath{\ldots}})_2
\]
where
\begin{align*}
x_- &:= {\rm fl}^{\rm down}(x) = 2^{\green{q}-\ensuremath{\sigma}} (1.\blue{b_1b_2\ensuremath{\ldots}b_S})_2 \\
x_+ &:= {\rm fl}^{\rm up}(x) = x_- + 2^{\green{q}-\ensuremath{\sigma}-S}
\end{align*}
Write the half-way point as:
\[
x_{\rm h} := {x_+ + x_- \over 2} = x_- + 2^{\green{q}-\ensuremath{\sigma}-S-1} = 2^{\green{q}-\ensuremath{\sigma}} (1.\blue{b_1b_2\ensuremath{\ldots}b_S}\red{1})_2
\]
If $x_- \ensuremath{\leq} x < x_{\rm h}$ then ${\rm fl}(x) = x_-$ and if $x_{\rm h} < x \ensuremath{\leq} x_+$ then ${\rm fl}(x) = x_{\rm h}$. If $x = x_{\rm h}$ then it is exactly half-way between $x_-$ and $x_+$. The rule is if $b_S = 0$ then ${\rm fl}(x) = x_-$ and otherwise ${\rm fl}(x) = x_+$.

In IEEE arithmetic, the arithmetic operations \texttt{+}, \texttt{-}, \texttt{*}, \texttt{/} are defined by the property that they are exact up to rounding.  Mathematically we denote these operations as $\ensuremath{\oplus}, \ensuremath{\ominus}, \ensuremath{\otimes}, \ensuremath{\oslash} : F_{\ensuremath{\sigma},Q,S} \ensuremath{\otimes} F_{\ensuremath{\sigma},Q,S} \ensuremath{\rightarrow} F_{\ensuremath{\sigma},Q,S}$ as follows:
\begin{align*}
x \ensuremath{\oplus} y &:= {\rm fl}(x+y) \\
x \ensuremath{\ominus} y &:= {\rm fl}(x - y) \\
x \ensuremath{\otimes} y &:= {\rm fl}(x * y) \\
x \ensuremath{\oslash} y &:= {\rm fl}(x / y)
\end{align*}
Note also that  \texttt{\^{}} and \texttt{sqrt} are similarly exact up to rounding. Also, note that when we convert a Julia command with constants specified by decimal expansions we first round the constants to floats, e.g., \texttt{1.1 + 0.1} is actually reduced to
\[
{\rm fl}(1.1) \ensuremath{\oplus} {\rm fl}(0.1)
\]
This includes the case where the constants are integers (which are normally exactly floats but may be rounded if extremely large).

\begin{example}[decimal is not exact] On a computer \texttt{1.1+0.1} is close to but not exactly the same thing as \texttt{1.2}. This is because ${\rm fl}(1.1) \ensuremath{\neq} 1+1/10$ and ${\rm fl}(0.1) \ensuremath{\neq} 1/10$ since their expansion in \emph{binary} is not finite. For $F_{16}$ we have:
\begin{align*}
{\rm fl}(1.1) &= {\rm fl}((1.0001100110\red{011\ensuremath{\ldots}})_2) =  (1.0001100110)_2 \\
{\rm fl}(0.1) &= {\rm fl}(2^{-4}(1.1001100110\red{011\ensuremath{\ldots}})_2) =  2^{-4} * (1.1001100110)_2 = (0.00011001100110)_2
\end{align*}
Thus when we add them we get
\[
{\rm fl}(1.1) + {\rm fl}(0.1) = (1.0011001100\red{011})_2
\]
where the red digits indicate those beyond the 10 significant digits representable in $F_{16}$. In this case we round down and get
\[
{\rm fl}(1.1) \ensuremath{\oplus} {\rm fl}(0.1) = (1.0011001100)_2
\]
On the other hand,
\[
{\rm fl}(1.2) = {\rm fl}((1.0011001100\red{11001100\ensuremath{\ldots}})_2) = (1.0011001101)_2
\]
which differs by 1 bit. \end{example}

\textbf{WARNING (non-associative)} These operations are not associative! E.g. $(x \ensuremath{\oplus} y) \ensuremath{\oplus} z$ is not necessarily equal to $x \ensuremath{\oplus} (y \ensuremath{\oplus} z)$. Commutativity is preserved, at least.

\subsection{Bounding errors in floating point arithmetic}
When dealing with normal numbers there are some important constants that we will use to bound errors.

\begin{definition}[machine epsilon/smallest positive normal number/largest normal number] \emph{Machine epsilon} is denoted
\[
\ensuremath{\epsilon}_{{\rm m},S} := 2^{-S}.
\]
When $S$ is implied by context we use the notation $\ensuremath{\epsilon}_{\rm m}$. The \emph{smallest positive normal number} is $q = 1$ and $b_k$ all zero:
\[
\min |F_{\ensuremath{\sigma},Q,S}^{\rm normal}| = 2^{1-\ensuremath{\sigma}}
\]
where $|A| := \{|x| : x \in A \}$. The \emph{largest (positive) normal number} is
\[
\max F_{\ensuremath{\sigma},Q,S}^{\rm normal} = 2^{2^Q-2-\ensuremath{\sigma}} (1.11\ensuremath{\ldots})_2 = 2^{2^Q-2-\ensuremath{\sigma}} (2-\ensuremath{\epsilon}_{\rm m})
\]
\end{definition}

We can bound the error of basic arithmetic operations in terms of machine epsilon, provided a real number is close to a normal number:

\begin{definition}[normalised range] The \emph{normalised range} ${\cal N}_{\ensuremath{\sigma},Q,S} \ensuremath{\subset} \ensuremath{\bbR}$ is the subset of real numbers that lies between the smallest and largest normal floating-point number:
\[
{\cal N}_{\ensuremath{\sigma},Q,S} := \{x : \min |F_{\ensuremath{\sigma},Q,S}^{\rm normal}| \ensuremath{\leq} |x| \ensuremath{\leq} \max F_{\ensuremath{\sigma},Q,S}^{\rm normal} \}
\]
When $\ensuremath{\sigma},Q,S$ are implied by context we use the notation ${\cal N}$. \end{definition}

We can use machine epsilon to determine bounds on rounding:

\begin{proposition}[round bound] If $x \in {\cal N}$ then
\[
{\rm fl}^{\rm mode}(x) = x (1 + \ensuremath{\delta}_x^{\rm mode})
\]
where the \emph{relative error} is bounded by:
\begin{align*}
|\ensuremath{\delta}_x^{\rm nearest}| &\ensuremath{\leq} {\ensuremath{\epsilon}_{\rm m} \over 2} \\
|\ensuremath{\delta}_x^{\rm up/down}| &< {\ensuremath{\epsilon}_{\rm m}}.
\end{align*}
\end{proposition}
\textbf{Proof}

We will show this result for the nearest rounding mode. Note first that
\[
{\rm fl}(-x) = -{\rm fl}(x)
\]
and hence it suffices to prove the result for positive $x$. Write
\[
x = 2^{\green{q}-\ensuremath{\sigma}} (1.b_1b_2\ensuremath{\ldots}b_S\red{b_{S+1}\ensuremath{\ldots}})_2.
\]
Define
\begin{align*}
x_- &:= {\rm fl}^{\rm down}(x) = 2^{\green{q}-\ensuremath{\sigma}} (1.b_1b_2\ensuremath{\ldots}b_S)_2 \\
x_+ &:= {\rm fl}^{\rm up}(x) = x_- + 2^{\green{q}-\ensuremath{\sigma}-S} \\
x_{\rm h} &:= {x_+ + x_- \over 2} = x_- + 2^{\green{q}-\ensuremath{\sigma}-S-1} = 2^{\green{q}-\ensuremath{\sigma}} (1.b_1b_2\ensuremath{\ldots}b_S\red{1})_2
\end{align*}
so that $x_- \ensuremath{\leq} x \ensuremath{\leq} x_+$. We consider two cases separately.

(\textbf{Round Down}) First consider the case where $x$ is such that we round down: ${\rm fl}(x) = x_-$. Since $2^{\green{q}-\ensuremath{\sigma}} \ensuremath{\leq} x_- \ensuremath{\leq} x \ensuremath{\leq} x_{\rm h}$ we have
\[
|\ensuremath{\delta}_x| = {x - x_- \over x} \ensuremath{\leq} {x_{\rm h} - x_- \over x_-} = {2^{\green{q}-\ensuremath{\sigma}-S-1} \over 2^{\green{q}-\ensuremath{\sigma}}} = 2^{-S-1} = {\ensuremath{\epsilon}_{\rm m} \over 2}.
\]
(\textbf{Round Up}) If ${\rm fl}(x) = x_+$ then $2^{\green{q}-\ensuremath{\sigma}} \ensuremath{\leq} x_- < x_{\rm h} \ensuremath{\leq} x \ensuremath{\leq} x_+$ and hence
\[
|\ensuremath{\delta}_x| = {x_+ - x \over x} \ensuremath{\leq} {x_+ - x_{\rm h} \over x_-} = {2^{\green{q}-\ensuremath{\sigma}-S-1} \over 2^{\green{q}-\ensuremath{\sigma}}} = 2^{-S-1} = {\ensuremath{\epsilon}_{\rm m} \over 2}.
\]
\ensuremath{\QED}

This immediately implies relative error bounds on all IEEE arithmetic operations, e.g., if $x+y \in {\cal N}$ then we have
\[
x \ensuremath{\oplus} y = (x+y) (1 + \ensuremath{\delta}_1)
\]
where (assuming the default nearest rounding) $|\ensuremath{\delta}_1| \ensuremath{\leq} {\ensuremath{\epsilon}_{\rm m} \over 2}.$

\subsection{Idealised floating point}
With a complicated formula it is mathematically inelegant to work with normalised ranges: one cannot guarantee apriori that a computation always results in a normal float. Extending the bounds to subnormal numbers is tedious, rarely relevant, and beyond the scope of this module. Thus to avoid this issue we will work with an alternative mathematical model:

\begin{definition}[idealised floating point] An idealised mathematical model of floating point numbers for which the only subnormal number is zero can be defined as:
\[
F_{\ensuremath{\infty},S} := \{\ensuremath{\pm} 2^q \ensuremath{\times} (1.b_1b_2b_3\ensuremath{\ldots}b_S)_2 :  q \ensuremath{\in} \ensuremath{\bbZ} \} \ensuremath{\cup} \{0\}
\]
\end{definition}

Note that $F^{\rm normal}_{\ensuremath{\sigma},Q,S} \ensuremath{\subset} F_{\ensuremath{\infty},S}$ for all $\ensuremath{\sigma},Q \ensuremath{\in} \ensuremath{\bbN}$. The definition of rounding ${\rm fl}_{\ensuremath{\infty},S}^{mode} : \ensuremath{\bbR} \ensuremath{\rightarrow} F_{\ensuremath{\infty},S}$ naturally extend to $F_{\ensuremath{\infty},S}$ and hence we can consider bounds for floating point operations such as $\ensuremath{\oplus}$, $\ensuremath{\ominus}$, etc. And in this model the round bound is valid for all real numbers (including $x = 0$).

\begin{example}[bounding a simple computation] We show how to bound the error in computing $(1.1 + 1.2) * 1.3 = 2.99$ and we may assume idealised floating-point arithmetic $F_{\ensuremath{\infty},S}$. First note that \texttt{1.1} on a computer is in fact ${\rm fl}(1.1)$, and we will always assume nearest rounding unless otherwise stated. Thus this computation becomes
\[
({\rm fl}(1.1) \ensuremath{\oplus} {\rm fl}(1.2)) \ensuremath{\otimes} {\rm fl}(1.3)
\]
We will show the \emph{absolute error} is given by
\[
({\rm fl}(1.1) \ensuremath{\oplus} {\rm fl}(1.2)) \ensuremath{\otimes} {\rm fl}(1.3) = 2.99 + \ensuremath{\delta}
\]
where $|\ensuremath{\delta}| \ensuremath{\leq}  23 \ensuremath{\epsilon}_{\rm m}.$ First we find
\meeq{
{\rm fl}(1.1) \ensuremath{\oplus} {\rm fl}(1.2) = (1.1(1 + \ensuremath{\delta}_1) + 1.2 (1+\ensuremath{\delta}_2))(1 + \ensuremath{\delta}_3) \ccr
 = 2.3 + \underbrace{1.1 \ensuremath{\delta}_1 + 1.2 \ensuremath{\delta}_2 + 2.3 \ensuremath{\delta}_3 + 1.1 \ensuremath{\delta}_1 \ensuremath{\delta}_3 + 1.2 \ensuremath{\delta}_2 \ensuremath{\delta}_3}_{\ensuremath{\varepsilon}_1}.
}
While $\ensuremath{\delta}_1 \ensuremath{\delta}_3$ and $\ensuremath{\delta}_2 \ensuremath{\delta}_3$ are absolutely tiny in practice we will bound them rather naïvely by eg.
\[
|\ensuremath{\delta}_1 \ensuremath{\delta}_3| \ensuremath{\leq} \ensuremath{\epsilon}_{\rm m}^2/4 \ensuremath{\leq} \ensuremath{\epsilon}_{\rm m}/4.
\]
Further we round up constants to integers in the bounds for simplicity. We thus have the bound
\[
|\ensuremath{\varepsilon}_1| \ensuremath{\leq} (2+2+3+1+1) {\ensuremath{\epsilon}_{\rm m} \over 2} \ensuremath{\leq} 5\ensuremath{\epsilon}_{\rm m}.
\]
Writing ${\rm fl}(1.3) = 1.3 (1+\ensuremath{\delta}_4)$ and also incorporating an error from the rounding in $\ensuremath{\otimes}$ we arrive at
\meeq{
({\rm fl}(1.1) \ensuremath{\oplus} {\rm fl}(1.2)) \ensuremath{\otimes} {\rm fl}(1.3) =
                (2.3 + \ensuremath{\varepsilon}_1) 1.3 (1 + \ensuremath{\delta}_4) (1 + \ensuremath{\delta}_5) \ccr
                 = 2.99 + \underbrace{1.3( \ensuremath{\varepsilon}_1 + 2.3\ensuremath{\delta}_4 + 2.3\ensuremath{\delta}_5 + \ensuremath{\varepsilon}_1 \ensuremath{\delta}_4 + \ensuremath{\varepsilon}_1 \ensuremath{\delta}_5 + 2.3 \ensuremath{\delta}_4 \ensuremath{\delta}_5 + \ensuremath{\varepsilon}_1\ensuremath{\delta}_4\ensuremath{\delta}_5)}_\ensuremath{\delta}
}
We use the bounds
\begin{align*}
|\ensuremath{\varepsilon}_1 \ensuremath{\delta}_4|, |\ensuremath{\varepsilon}_1 \ensuremath{\delta}_5| &\ensuremath{\leq} 5 \ensuremath{\epsilon}_{\rm m}^2/2 \ensuremath{\leq} 5 \ensuremath{\epsilon}_{\rm m}/2,  \cr
|\ensuremath{\delta}_4 \ensuremath{\delta}_5| &\ensuremath{\leq}  \ensuremath{\epsilon}_{\rm m}^2/4  \ensuremath{\leq} \ensuremath{\epsilon}_{\rm m}/4, \cr
|\ensuremath{\varepsilon}_1\ensuremath{\delta}_4\ensuremath{\delta}_5| &\ensuremath{\leq} 5\ensuremath{\epsilon}_{\rm m}^3/4 \ensuremath{\leq} 5\ensuremath{\epsilon}_{\rm m}/4.
\end{align*}
Thus the \emph{absolute error} is bounded (bounding 1.3 by $3/2$) by
\[
|\ensuremath{\delta}| \ensuremath{\leq} (3/2) (5 +  3/2 + 3/2 + 5/2 + 5/2 + 3/4 + 5/4) \ensuremath{\epsilon}_{\rm m} \ensuremath{\leq} 23 \ensuremath{\epsilon}_{\rm m}.
\]
\end{example}

\subsection{Divided differences floating point error bound}
We can use the bound on floating point arithmetic to deduce a bound on divided differences that captures the phenomena we observed where the error of divided differences became large as $h \ensuremath{\rightarrow} 0$. We assume that the function we are attempting to differentiate is computed using floating point arithmetic in a way that has a small absolute error.

\begin{theorem}[divided difference error bound] Assume we are working in idealised floating-point arithmetic $F_{\ensuremath{\infty},S}$. Let $f$ be twice-differentiable in a neighbourhood of $x \ensuremath{\in} F_{\ensuremath{\infty},S}$ and assume that
\[
 f(x) = f^{\rm FP}(x) + \ensuremath{\delta}_x^f
\]
where $f^{\rm FP} : F_{S,\ensuremath{\infty}} \ensuremath{\rightarrow} F_{S,\ensuremath{\infty}}$ has uniform absolute accuracy in that neighbourhood, that is:
\[
|\ensuremath{\delta}_x^f| \ensuremath{\leq} c \ensuremath{\epsilon}_{\rm m}
\]
for a fixed constant $c \ensuremath{\geq} 0$. The divided difference approximation partially implemented with floating point satisfies
\[
{f^{\rm FP}(x + h) \ensuremath{\ominus} f^{\rm FP}(x) \over h} = f'(x) + \ensuremath{\delta}_{x,h}^{\rm FD}
\]
where
\[
|\ensuremath{\delta}_{x,h}^{\rm FD}| \ensuremath{\leq} {|f'(x)| \over 2} \ensuremath{\epsilon}_{\rm m} + M h +  {4c \ensuremath{\epsilon}_{\rm m} \over h}
\]
for $M = \sup_{x \ensuremath{\leq} t \ensuremath{\leq} x+h} |f''(t)|$.

\end{theorem}
\textbf{Proof}

We have
\begin{align*}
(f^{\rm FP}(x + h) \ensuremath{\ominus} f^{\rm FP}(x)) / h &= {f(x + h) -  \ensuremath{\delta}^f_{x+h} - f(x) + \ensuremath{\delta}^f_x \over h} (1 + \ensuremath{\delta}_1) \\
&= {f(x+h) - f(x) \over h} (1 + \ensuremath{\delta}_1) + { \ensuremath{\delta}^f_x - \ensuremath{\delta}^f_{x+h} \over h} (1 + \ensuremath{\delta}_1)
\end{align*}
where $|\ensuremath{\delta}_1| \ensuremath{\leq} {\ensuremath{\epsilon}_{\rm m} / 2}$. Applying Taylor's theorem we get
\[
(f^{\rm FP}(x + h) \ensuremath{\ominus} f^{\rm FP}(x)) / h = f'(x) + \underbrace{f'(x) \ensuremath{\delta}_1 + {f''(t) \over 2} h (1 + \delta_1) + {\ensuremath{\delta}^f_x  - \ensuremath{\delta}^f_{x+h}\over h} (1 + \ensuremath{\delta}_1)}_{\ensuremath{\delta}_{x,h}^{\rm FD}}
\]
The bound then follows, using the very pessimistic bound $|1 + \ensuremath{\delta}_1| \ensuremath{\leq} 2$.

\ensuremath{\QED}

The previous theorem neglected some errors due to rounding, which was done for simplicity. This is justified under fairly general restrictions:

\begin{corollary}[divided differences in practice] We have
\[
(f^{\rm FP}(x \ensuremath{\oplus} h) \ensuremath{\ominus} f^{\rm FP}(x)) \ensuremath{\oslash} h = {f^{\rm FP}(x + h) \ensuremath{\ominus} f^{\rm FP}(x) \over h}
\]
whenever  $h = 2^{j-n}$ for $0 \ensuremath{\leq} n \ensuremath{\leq} S$ and the last binary place of $x \ensuremath{\in} F_{\ensuremath{\infty},S}$ is zero, that is $x = \ensuremath{\pm}2^j (1.b_1\ensuremath{\ldots}b_{S-1}0)_2$.

\end{corollary}
\textbf{Proof}

When $h = 2^{-n}$ we have for any normal float $y = \ensuremath{\pm}2^j (1.b_1\ensuremath{\ldots}b_S)_2 \ensuremath{\in} F_{\ensuremath{\infty},S}$ that 
\[
y/h = \ensuremath{\pm}2^{j-n} (1.b_1\ensuremath{\ldots}b_S)_2 \ensuremath{\in} F_{\ensuremath{\infty},S} \ensuremath{\Rightarrow} y/h = y \ensuremath{\oslash} h.
\]
If $b_S = 0$ the worst possible case is that we increase the exponent by one as we are just adding $1$ to one of the digits $b_1,\ensuremath{\ldots},b_S$. This would cause us to lose the last digit. But if that is zero no error is incurred when we round.

\ensuremath{\QED}

The three-terms of this bound tell us a story: the first term is a fixed (small) error, the second term tends to zero as $h \rightarrow 0$, while the last term grows like $\ensuremath{\epsilon}_{\rm m}/h$ as $h \rightarrow 0$.  Thus we observe convergence while the second term dominates, until the last term takes over. Of course, a bad upper bound is not the same as a proof that something grows, but it is a good indication of what happens \emph{in general} and suffices to choose $h$ so that these errors are balanced (and thus minimised). Since in general we do not have access to the constants $c$ and $M$ we employ the following heuristic to balance the two sources of errors:

\textbf{Heuristic (divided difference with floating-point step)} Choose $h$ proportional to $\sqrt{\ensuremath{\epsilon}_{\rm m}}$ in divided differences  so that $M h$ and ${4c \ensuremath{\epsilon}_{\rm m} \over h}$ are (roughly) the same magnitude.

In the case of double precision $\sqrt{\ensuremath{\epsilon}_{\rm m}} \ensuremath{\approx} 1.5\ensuremath{\times} 10^{-8}$, which is close to when the observed error begins to increase in the examples we saw before.

\textbf{Remark} While divided differences is of debatable utility for computing derivatives, it is extremely effective in building methods for solving differential equations, as we shall see later. It is also very useful as a \ensuremath{\ldq}sanity check" if one wants something to compare with other numerical methods for differentiation.

\textbf{Remark} It is also possible to deduce an error bound for the rectangular rule showing that the error caused by round-off is on the order of $n \ensuremath{\epsilon}_{\rm m}$, that is it does in fact grow but the error without round-off which was bounded by $M/n$ will be substantially greater for all reasonable values of $n$.




\input{II.4.Intervals}


\chapter{Numerical Linear Algebra}

Many problems in mathematics are linear: for example, polynomial regression and
differential equations. Numerical methods for such applications invariably result
in (finite-dimensional) linear systems that must be solved numerically on a computer: 
the dimensions of the problems are often in the 1000s, millions, or even billions.
One would certainly not want to tackle that with Gaussian elimination by hand!
In this chapter we discuss algorithms, and in particular matrix factorisations, that are
computed using floating point operations. We also introduce some basic applications.



In particular we discuss:

\begin{enumerate}
    \item III.1 Structured Matrices: we discuss special structured matrices such as triangular and tridiagonal.
    \item III.2 Differential Equations: using divided differences we can reduce differential equations
    to linear systems. This motivates the investigation of numerical algorithms for solving linear systems.
    \item III.3 LU and Cholesky Factorisations: we look at computing a factorisation of a square matrix as a product of a lower and upper triangular matrix, including the special case where the matrix is symmetric positive
    definite. Hidden in this is an algorithm to prove positive definiteness.
\item III.4 Polynomial Regression: often in data science one needs to approximate data by a polynomial.
We discuss how to reduce this problem to solving a rectangular least squares problem.
\item III.5 Orthogonal Matrices: we discuss different types of orthogonal matrices, which will be used to simplify rectangular least squares problems.
\item III.6 QR Factorisation: we introduce an algorithm to compute a factorisation of a rectangular matrix as a product of an orthogonal and upper triangular matrix, thereby solving least squares problems.
\end{enumerate}


\section{Structured Matrices}
We have seen how algebraic operations (\texttt{+}, \texttt{-}, \texttt{*}, \texttt{/}) are defined exactly in terms of rounding ($\ensuremath{\oplus}$, $\ensuremath{\ominus}$, $\ensuremath{\otimes}$, $\ensuremath{\oslash}$) for floating point numbers. Now we see how this allows us to do (approximate) linear algebra operations on matrices.

A matrix can be stored in different formats, in particular it is important for large scale simulations that we take advantage of \emph{sparsity}: if we know a matrix has entries that are guaranteed to be zero we can implement faster algorithms. We shall see that this comes up naturally in numerical methods for solving differential equations.

In particular, we will discuss some basic types of structure in matrices:

\begin{itemize}
\item[1. ] \emph{Dense}: This can be considered unstructured, where we need to store all entries in a vector or matrix. Matrix-vector multiplication reduces directly to standard algebraic operations. Solving linear systems with dense matrices will be discussed later.


\item[2. ] \emph{Triangular}: If a matrix is upper or lower triangular, multiplication requires roughly half the number of operations. Crucially, we can apply the inverse of a triangular matrix using forward- or back-substitution.


\item[3. ] \emph{Banded}: If a matrix is zero apart from entries a fixed distance from  the diagonal it is called banded and matrix-vector multiplication has a lower \emph{complexity}: the number of operations scales linearly with the dimension (instead of quadratically). We discuss three cases: diagonal, tridiagonal and bidiagonal matrices.

\end{itemize}
\textbf{Remark} For those who took the first half of the module, there was an important emphasis on working with \emph{linear operators} rather than \emph{matrices}. That is, there was an emphasis on basis-independent mathematical techniques, which is critical for extension of results to infinite-dimensional spaces (which might not have a complete basis). However, in terms of practical computation we need to work with some representation of an operator and the most natural is a matrix. And indeed we will see in the next section how infinite-dimensional differential equations can be solved by reduction to finite-dimensional matrices. (Restricting attention to matrices is also important as some of the students have not taken the first half of the module.)

\subsection{Dense matrices}
A basic operation is matrix-vector multiplication. For a field $\ensuremath{\bbF}$ (typically $\ensuremath{\bbR}$ or $\ensuremath{\bbC}$, or this can be relaxed to be a ring), consider a matrix and vector whose entries are in $\ensuremath{\bbF}$:
\[
A = \begin{bmatrix}
a_{11} & \ensuremath{\cdots} & a_{1n} \\
\ensuremath{\vdots} & \ensuremath{\ddots} & \ensuremath{\vdots} \\
a_{m1} & \ensuremath{\cdots} & a_{mn}
\end{bmatrix} = \begin{bmatrix} \ensuremath{\bm{\a}}_1 | \ensuremath{\cdots} | \ensuremath{\bm{\a}}_n \end{bmatrix} \ensuremath{\in} \ensuremath{\bbF}^{m \ensuremath{\times} n}, \qquad
\ensuremath{\bm{\x}} = \Vectt[x_1,\ensuremath{\vdots},x_n] \ensuremath{\in} \ensuremath{\bbF}^n.
\]
where $\ensuremath{\bm{\a}}_j = A \ensuremath{\bm{\e}}_j \ensuremath{\in} \ensuremath{\bbF}^m$ are the columns of $A$. Recall the usual definition of matrix multiplication:
\[
A\ensuremath{\bm{\x}} := \begin{bmatrix} \ensuremath{\sum}_{j=1}^n a_{1j} x_j \\ \ensuremath{\vdots} \\ \ensuremath{\sum}_{j=1}^n a_{mj} x_j \end{bmatrix}.
\]
When we are working with floating point numbers $A \ensuremath{\in} F^{m \ensuremath{\times} n}$ we obtain an approximation:
\[
A\ensuremath{\bm{\x}} \ensuremath{\approx} \begin{bmatrix} \ensuremath{\bigoplus}_{j=1}^n (a_{1j}  \ensuremath{\otimes} x_j) \\ \ensuremath{\vdots} \\  \ensuremath{\bigoplus}_{j=1}^n (a_{mj}  \ensuremath{\otimes} x_j) \end{bmatrix}.
\]
This actually encodes an algorithm for computing the entries.

This algorithm uses $O(m n)$ floating point operations (see the appendix if you are unaware of Big-O notation, here our complexities are implicitely taken to be when $m$ or $n$ tends to $\ensuremath{\infty}$): each of the $m$ entry consists of $n$ multiplications and $n-1$ additions, hence we have a total of $2n-1 = O(n)$ operations per row for a total of $m(2n-1) = O(mn)$ operations. For a square matrix this is $O(n^2)$ operations which we call \emph{quadratic complexity}. In the problem sheet we see how the floating point error can be bounded in terms of norms, thus reducing the problem to a purely mathematical concept.

Sometimes there are multiple ways of implementing numerical algorithms. We have an alternative formula where we multiply by columns:
\[
A \ensuremath{\bm{\x}} = x_1 \ensuremath{\bm{\a}}_1  + \ensuremath{\cdots} + x_n \ensuremath{\bm{\a}}_n.
\]
The floating point formula for this is exactly the same as the previous algorithm and the number of operations is the same. Just the order of operations has changed. Suprisingly, this latter version is significantly faster.

\textbf{Remark} Floating point operations are sometimes called FLOPs, which are a standard measurement  of speed of CPUs. However, FLOP sometimes uses an alternative definitions that combines an addition and multiplication as a single FLOP. In the lab we give an example showing that counting the precise number of operations is somewhat of a fools errand: algorithms such as the two approaches for matrix multiplication with the exact same number of operations can have wildly different speeds. We will therefore only be concerned with \emph{complexity}; the asymptotic growth (Big-O) of operations as $n \ensuremath{\rightarrow} \ensuremath{\infty}$, in which case the difference between FLOPs and operations is immaterial.

\subsection{Triangular matrices}
The simplest sparsity case is being triangular: where all entries above or below the diagonal are zero. We consider upper and lower triangular matrices:
\[
U = \begin{bmatrix}
u_{11} & \ensuremath{\cdots} & u_{1n} \\
 & \ensuremath{\ddots} & \ensuremath{\vdots} \\
 &  & u_{nn}
\end{bmatrix}, \qquad L = \begin{bmatrix}
\ensuremath{\ell}_{11} &  \\
\ensuremath{\vdots} & \ensuremath{\ddots} & \\
\ensuremath{\ell}_{n1} & \ensuremath{\cdots} & \ensuremath{\ell}_{nn}
\end{bmatrix}.
\]
Matrix multiplication can be modified to take advantage of the zero pattern of the matrix. Eg., if $L \ensuremath{\in} \ensuremath{\bbF}^{n \ensuremath{\times} n}$ is lower triangular we have:
\[
L\ensuremath{\bm{\x}} = \begin{bmatrix} \ensuremath{\ell}_{1,1} x_1 \\ \ensuremath{\sum}_{j=1}^2 \ensuremath{\ell}_{2j} x_j  \\ \ensuremath{\vdots} \\ \ensuremath{\sum}_{j=1}^n \ensuremath{\ell}_{nj} x_j \end{bmatrix}.
\]
When implemented in floating point this uses roughly half the number of multiplications: $1 + 2 + \ensuremath{\ldots} + n = n(n+1)/2$ multiplications. (It is also about twice as fast in practice.) The complexity is still quadratic: $O(n^2)$ operations.

Triangularity allows us to also invert systems using forward- or back-substitution. In particular if $\ensuremath{\bm{\x}}$ solves $L \ensuremath{\bm{\x}} = \ensuremath{\bm{\b}}$ then we have:
\[
x_k = {b_k - \ensuremath{\sum}_{j=1}^{k-1} \ensuremath{\ell}_{kj} x_j \over \ensuremath{\ell}_{kk}}
\]
Thus we can compute $x_1,x_2,\ensuremath{\ldots},x_n$ in sequence.

\subsection{Banded matrices}
A \emph{banded matrix} is zero off a prescribed number of diagonals. We call the number of (potentially) non-zero diagonals the \emph{bandwidths}:

\begin{definition}[bandwidths] A matrix $A$ has \emph{lower-bandwidth} $l$ if $a_{kj} = 0$ for all $k-j > l$ and \emph{upper-bandwidth} $u$ if $a_{kj} = 0$ for all $j-k > u$. We say that it has \emph{strictly lower-bandwidth} $l$ if it has lower-bandwidth $l$ and there exists a $j$ such that $a_{j+l,j} \ensuremath{\neq} 0$. We say that it has \emph{strictly upper-bandwidth} $u$ if it has upper-bandwidth $u$ and there exists a $k$ such that $a_{k,k+u} \ensuremath{\neq} 0$. \end{definition}

A square banded matrix has the sparsity pattern:
\[
A = \begin{bmatrix}
a_{11} & \ensuremath{\cdots} & a_{1,u+1} \\
\ensuremath{\vdots} & a_{22} & \ensuremath{\ddots} &  a_{2,u+2} \\
a_{1+l,1} & \ensuremath{\ddots} & \ensuremath{\ddots} & \ensuremath{\ddots} & \ensuremath{\ddots} \\
& a_{2+l,2} & \ensuremath{\ddots} & \ensuremath{\ddots} &  \ensuremath{\ddots} & a_{n-u,n} \\
&& \ensuremath{\ddots} & \ensuremath{\ddots} & \ensuremath{\ddots} & \ensuremath{\vdots} \\
&&& a_{n,n-l} & \ensuremath{\cdots} & a_{nn}
\end{bmatrix}
\]
A banded matrix has better complexity for matrix multiplication and solving linear systems:  we can multiply square banded matrices in linear complexity: $O(n)$ operations. We consider two cases in particular (in addition to diagonal): bidiagonal and tridiagonal.

\begin{definition}[Bidiagonal] If a square matrix has bandwidths $(l,u) = (1,0)$ it is \emph{lower-bidiagonal} and if it has bandwidths $(l,u) = (0,1)$ it is \emph{upper-bidiagonal}. \end{definition}

For example, if
\[
L = \begin{bmatrix}\ensuremath{\ell}_{11} \\
\ensuremath{\ell}_{21}& \ensuremath{\ell}_{22} \\ 
& \ensuremath{\ddots} & \ensuremath{\ddots} \\
 &&\ensuremath{\ell}_{n,n-1} &\ensuremath{\ell}_{nn}
\end{bmatrix}
\]
then lower-bidiagonal multiplication becomes
\[
L\ensuremath{\bm{\x}} = \begin{bmatrix} \ensuremath{\ell}_{1,1} x_1 \\ \ensuremath{\ell}_{21} x_1 + \ensuremath{\ell}_{22} x_2    \\ \ensuremath{\vdots} \\ 
\ensuremath{\ell}_{n,n-1} x_{n-1} + \ensuremath{\ell}_{nn} x_n \end{bmatrix}.
\]
This requires $O(1)$ operations per row (at most 2 multiplications and 1 addition) and hence the total is only $O(n)$ operations. A bidiagonal matrix is always triangular and we can also invert in $O(n)$ operations: if $L \ensuremath{\bm{\x}} = \ensuremath{\bm{\b}}$ then $x_1 = b_1/\ensuremath{\ell}_{11}$  and for $k = 2,\ensuremath{\ldots},n$ we can compute
\[
x_k = {b_k - \ensuremath{\ell}_{k-1,k} x_{k-1} \over \ensuremath{\ell}_{kk}}.
\]
\begin{definition}[Tridiagonal] If a square matrix has bandwidths $l = u = 1$ it is \emph{tridiagonal}. \end{definition}

For example,
\[
A = \begin{bmatrix} a_{11} & a_{12} \\
a_{21} & a_{22} & a_{23} \\
 & \ensuremath{\ddots} & \ensuremath{\ddots} & \ensuremath{\ddots} \\
&& a_{n-1,n-2} &                                 a_{n-1,n-1} & a_{n-1,n} \\
&&&a_{n,n-1} & a_{nn}
\end{bmatrix}
\]
is tridiagonal. Matrix multiplication is clearly $O(n)$ operations: each row has $O(1)$ non-zeros and there are $n$ rows. But so is solving linear systems, which we shall see later.





\section{Differential Equations via Finite Differences}
Linear algebra is a powerful tool for solving linear equations, including \ensuremath{\infty}-dimensional ones like differential equations. In this section we discuss \emph{finite differences}: an algorithmic way of reducing ODEs to linear systems by replacing derivatives with divided difference approximations. 

We will focus on the following differential equations. Indefinite integration for $a \ensuremath{\leq} x \ensuremath{\leq} b$ can be viewed as solving a very simple first-order linear ODE: given a constant $c \ensuremath{\in} \ensuremath{\bbF}$ (where as before $\ensuremath{\bbF}$ is usually $\ensuremath{\bbR}$ or $\ensuremath{\bbC}$) and a function $f : [a,b] \ensuremath{\rightarrow} \ensuremath{\bbF}$, find a differentiable function $u : [a,b] \ensuremath{\rightarrow} \ensuremath{\bbF}$ such that
\meeq{
u(a) = c, \ccr
u'(x) = f(x).
}
We will then allow for more complicated first order linear ODEs with variable coefficients: given  a constant $c$ and functions $f,\ensuremath{\omega} : [a,b] \ensuremath{\rightarrow} \ensuremath{\bbF}$ find $u$ such that
\meeq{
u(a) = c, \ccr 
u'(x) - \ensuremath{\omega}(x) u(x) = f(x).
}
For second-order differential equations you may have seen \emph{initial value problems} where the value and derivative at an initial point $x=a$ are provided.  Instead, we will consider \emph{boundary value problems} where the value at the left and right endpoints are imposed. In particular we will consider the Poisson equation with \emph{Dirichlet conditions} (i.e. conditions on the left and right of an interval): given constants $c,d$ and a function $f$ find a twice-differentiable function $u$ such that
\meeq{
u(a) = c, \ccr
u''(x) = f(x), \ccr
u(b) = d
}
In higher dimensions, the Poisson equation (and other \emph{elliptic} partial differential equations) typically have boundary conditions imposed on the boundary of a complicated geometry  and the techniques we discuss for our simple 1D model problem extend to these more challenging settings.

Briefly, the basic idea of finite differences is a systematic way of reducing a differential equation to a linear system. For each problem we will do the following steps:

\begin{itemize}
\item[1. ] Discretise $[a,b]$ by a grid of points $x_0,\ensuremath{\ldots},x_n$ and write the ODE on each grid point.


\item[2. ] Replace derivatives of the solution $u$ with its values on a grid $u(x_j)$ by using a divided difference formula.


\item[3. ] In the formula replace unknown values of $u(x_j)$ at the grid by new unknowns $u_j$.


\item[4. ] Deduce from this  a linear system that can be solved to compute $u_j$ so that (hopefully) $u_j \ensuremath{\approx} u(x_j)$. 

\end{itemize}
\textbf{Remark} One can prove convergence of finite difference approximations but this is beyond the scope of this module. In the lab we determine convergence rates experimentally.

\subsection{Indefinite integration}
We begin with the simplest differential equation on an interval $[a,b]$:
\begin{align*}
u(a) &= c, \\
u'(x) &= f(x)
\end{align*}
As in integration we will use an evenly spaced grid $a = x_0 < x_1 < \ensuremath{\ldots} < x_n = b$ defined by $x_j :=  a + h j$ where $h := (b-a)/n$. The solution is of course $u(x) = c + \ensuremath{\int}_a^x f(x) {\rm d}x$ and we could use Rectangular or Trapezium rules to to obtain approximations to $u(x_j)$ for each $j$, however, we shall take another approach that will generalise to other differential equations. 

Consider a divided difference approximation like right-sided divided differences: 
\[
u'(x) \ensuremath{\approx} {u(x+h) - u(x)\over h}.
\]
When applied to a grid point $x_j \ensuremath{\in} \set{x_0,\ensuremath{\ldots},x_{n-1}}$ this becomes:
\[
u'(x_j) \ensuremath{\approx} {u(x_j+h) - u(x_j)\over h} = {u(x_{j+1}) - u(x_j)\over h}
\]
Note that $x_n$ is not permitted since that would depend on $u(x_{n+1})$, but $x_{n+1} > b$ and we have only assumed $f$ is defined on $[a,b]$. We use this approximation as follows:

(1) Write the ODE and initial conditions on the grid. Since right-sided divided differences will depend on $x_j$ and $x_{j+1}$ we stop at $x_{n-1}$ to avoid going past our grid: 
\[
\Vectt[u(x_0) , 
     u'(x_0) ,
u'(x_1) ,
\ensuremath{\vdots} ,
u'(x_{n-1})] = \underbrace{\Vectt[c, f(x_0), f(x_1), \ensuremath{\vdots} , f(x_{n-1})]}_{\ensuremath{\bm{\b}}}
\]
(2) Replace derivatives with divided differences, changing equality to an approximation:
\[
\Vectt[u(x_0) \\ 
(u(x_1) - u(x_0))/h \\
(u(x_2) - u(x_1))/h \\
\ensuremath{\vdots} \\
(u(x_n) - u(x_{n-1})/h] \ensuremath{\approx} \ensuremath{\bm{\b}}
\]
(3) We do not know $u(x_j)$ hence we replace it with other unknowns $u_j$, but where the approximation is turned back into an equality: we want to find $u_0,\ensuremath{\ldots},u_n$ such that
\[
\Vectt[u_0 \\ 
(u_1 - u_0)/h \\
(u_2 - u_1)/h \\
\ensuremath{\vdots} \\
(u_n - u_{n-1})/h] = \ensuremath{\bm{\b}}
\]
(4) This can be rewritten as a lower bidiagonal linear system:
\[
\underbrace{\begin{bmatrix}
    1 \\ 
    -1/h & 1/h \\
    & \ensuremath{\ddots} & \ensuremath{\ddots} \\
    && -1/h & 1/h \end{bmatrix}}_L \underbrace{\Vectt[u_0,u_1,\ensuremath{\vdots},u_n]}_{\ensuremath{\bm{\u}}} = \ensuremath{\bm{\b}}
\]
We can determine $\ensuremath{\bm{\u}}$ by solving $L \ensuremath{\bm{\u}} = \ensuremath{\bm{\b}}$ using forward-substitution.

As mentioned before, it is possible to prove that as $n \ensuremath{\rightarrow} \ensuremath{\infty}$ we have for all $j$ that $u_j \ensuremath{\rightarrow} u(x_j)$ (uniformly). But we will leave this to be shown experimentally in the lab.

\subsection{Forward Euler}
We can extend this to more general first-order linear differential equations with a variable coefficient:
\begin{align*}
u(a) &= c \\
u'(x) + \ensuremath{\omega}(x) u(x) &= f(x)
\end{align*}
The steps proceed very similar to before:

(1) Write the ODE and initial conditions on the grid, avoiding $x_n$ so that we don't go past the endpoint:
\[
\Vectt[u(x_0) \\ 
u'(x_0) + \ensuremath{\omega}(x_0) u(x_0) \\
u'(x_1) + \ensuremath{\omega}(x_1) u(x_1) \\
\ensuremath{\vdots} \\
u'(x_{n-1})+ \ensuremath{\omega}(x_{n-1}) u(x_{n-1})] = \underbrace{\Vectt[c, f(x_0), f(x_1), \ensuremath{\vdots} , f(x_{n-1})]}_{\ensuremath{\bm{\b}}}
\]
(2) Replace derivatives with divided differences:
\[
\Vectt[u(x_0) \\ 
(u(x_1) - u(x_0))/h + \ensuremath{\omega}(x_0)u(x_0) \\
(u(x_2) - u(x_1))/h + \ensuremath{\omega}(x_1)u(x_1) \\
\ensuremath{\vdots} \\
(u(x_n) - u(x_{n-1}))/h + \ensuremath{\omega}(x_{n-1})u(x_{n-1})] \ensuremath{\approx} \ensuremath{\bm{\b}}
\]
(3) Replace $u(x_j)$  by its approximation $u_j$ but now with the system being an equality:
\[
\Vectt[u_0 \\ 
(u_1 - u_0)/h + \ensuremath{\omega}(x_0) u_0 \\
(u_2 - u_1)/h + \ensuremath{\omega}(x_1) u_1 \\
\ensuremath{\vdots} \\
(u_n - u_{n-1})/h  + \ensuremath{\omega}(x_{n-1}) u_{n-1}] = \ensuremath{\bm{\b}}
\]
(4) This is equivalent to the linear system:
\[
\underbrace{\begin{bmatrix}
    1 \\ 
    \ensuremath{\omega}(x_0)-1/h & 1/h \\
    & \ensuremath{\ddots} & \ensuremath{\ddots} \\
    && \ensuremath{\omega}(x_{n-1})-1/h & 1/h \end{bmatrix}}_L \underbrace{\Vectt[u_0,u_1,\ensuremath{\vdots},u_n]}_{\ensuremath{\bm{\u}}} = \ensuremath{\bm{\b}}
\]
We can solve $L \ensuremath{\bm{\u}} = \ensuremath{\bm{\b}}$ using forward-substitution so that $u_j \ensuremath{\approx} u(x_j)$.

\subsection{Poisson equation}
Consider the Poisson equation with Dirichlet conditions (a two-point boundary value problem): given constants $c,d$ and a function $f : [a,b] \ensuremath{\rightarrow} \ensuremath{\bbR}$ find a twice differentiable function $u : [a,b] \ensuremath{\rightarrow} \ensuremath{\bbR}$ satisfying
\begin{align*}
u(a) &= c, \\
u''(x) &= f(x), \\
u(b) &= d
\end{align*}
We shall adapt the procedure using the second-order divided difference approximation from the first probem sheet:
\[
u''(x) \ensuremath{\approx} {u(x-h) - 2u(x) + u(x+h)\over h^2}
\]
When applied to a grid point $x_1,\ensuremath{\ldots},x_{n-1}$ this becomes:
\[
u'(x_j) \ensuremath{\approx} {u(x_j-h) - 2u(x_j) + u(x_j+h)\over h^2} = {u(x_{j-1}) - 2u(x_j) + u(x_{j+1})\over h^2}
\]
Note that $x_0$ and $x_n$ are not permitted since that would go past the endpoints of the interval. We use this approximation as follows:

(1) Write the ODE and boundary conditions on the grid (putting the left condition on the top and right condition on the bottom):
\[
\Vectt[u(x_0) \\ 
u''(x_1) \\
u''(x_2) \\
\ensuremath{\vdots} \\
u''(x_{n-1}) \\
u(x_n)] = \underbrace{\Vectt[c, f(x_1), f(x_2), \ensuremath{\vdots} , f(x_{n-1}), d]}_{\ensuremath{\bm{\b}}}
\]
(2) Replace derivatives with divided differences:
\[
\Vectt[u(x_0) \\ 
{u(x_0) - 2u(x_1) + u(x_2)\over h^2} \\
{u(x_1) - 2u(x_2) + u(x_3)\over h^2} \\
\ensuremath{\vdots} \\
{u(x_{n-2}) - 2u(x_{n-1}) + u(x_n)\over h^2} \\
u(x_n)] \ensuremath{\approx} \ensuremath{\bm{\b}}
\]
(3) Replace $u(x_j)$  by its approximation $u_j$: we want to find $u_0,\ensuremath{\ldots},n_n$ so that
\[
\Vectt[u_0 \\ 
{u_0 - 2u_1 + u_2\over h^2} \\
{u_1 - 2u_2 + u_3\over h^2} \\
\ensuremath{\vdots} \\
{u_{n-2} - 2u_{n-1} + u_n\over h^2} \\
u_n] = \ensuremath{\bm{\b}}
\]
(4) This is equivalent to a tridiagonal linear system:
\[
\underbrace{\begin{bmatrix}
    1 \\ 
    1/h^2 & -2/h^2 & 1/h^2 \\
    & \ensuremath{\ddots} & \ensuremath{\ddots} & \ensuremath{\ddots} \\
   && 1/h^2 & -2/h^2 & 1/h^2 \\ 
   &&&& 1 \end{bmatrix}}_A \underbrace{\Vectt[u_0,u_1,\ensuremath{\vdots},u_n]}_{\ensuremath{\bm{\u}}} = \ensuremath{\bm{\b}}
\]
We are left with the question: how do we solve a tridiagonal linear system $A \ensuremath{\bm{\u}} = \ensuremath{\bm{\b}}$? We will answer this question in the next section.





\section{LU and Cholesky factorisations}
In this section we consider the following factorisations for square invertible  matrices $A$:

\begin{itemize}
\item[1. ] The \emph{LU factorisation}: $A = LU$ where $L$ is lower triangular and $U$ is upper triangular. This is equivalent to Gaussian elimination without pivoting, so may not exist (e.g. if $a_{11} = 0$).


\item[2. ] The \emph{PLU factorisation}: $A = P^\ensuremath{\top} LU$ where $P$ is a permutation matrix (a matrix when multiplying a vector is equivalent to permuting its rows), $L$ is lower triangular and $U$ is upper triangular. This is equivalent to Gaussian elimination with pivoting. It always exists but may be unstable in extremely rare cases. We won't discuss the details of computing the PLU factorisation but will explore practical usage in the lab.


\item[3. ] For a real square \emph{symmetric positive definite} ($A \ensuremath{\in} \ensuremath{\bbR}^{n \ensuremath{\times} n}$ such that $A^\ensuremath{\top} = A$ and $\ensuremath{\bm{\x}}^\ensuremath{\top} A \ensuremath{\bm{\x}} > 0$ for all $\ensuremath{\bm{\x}} \ensuremath{\in} \ensuremath{\bbR}^n$, $\ensuremath{\bm{\x}} \ensuremath{\neq} 0$)  matrix the LU decomposition has a special form called the \emph{Cholesky factorisation}: $A = L L^\ensuremath{\top}$. This provides an algorithmic way to \emph{prove} that a matrix is symmetric positive definite, and is roughly twice as fast as the LU factorisation to compute.

\end{itemize}
\subsection{Outer products}
In what follows we will use outer products extensively:

\begin{definition}[outer product] Given $\ensuremath{\bm{\x}} \ensuremath{\in} \ensuremath{\bbF}^m$ and $\ensuremath{\bm{\y}} \ensuremath{\in} \ensuremath{\bbF}^n$ the \emph{outer product} is:
\[
\ensuremath{\bm{\x}} \ensuremath{\bm{\y}}^\ensuremath{\top} := [\ensuremath{\bm{\x}} y_1 | \ensuremath{\cdots} | \ensuremath{\bm{\x}} y_n] = \begin{bmatrix} x_1 y_1 & \ensuremath{\cdots} & x_1 y_n \\
                        \ensuremath{\vdots} & \ensuremath{\ddots} & \ensuremath{\vdots} \\
                        x_m y_1 & \ensuremath{\cdots} & x_m y_n \end{bmatrix} \ensuremath{\in} \ensuremath{\bbF}^{m \ensuremath{\times} n}.
\]
Note this is equivalent to matrix-matrix multiplication if we view $\ensuremath{\bm{\x}}$ as a $m \ensuremath{\times} 1$ matrix and $\ensuremath{\bm{\y}}^\ensuremath{\top}$ as a $1 \ensuremath{\times} n$ matrix. \end{definition}

\begin{proposition}[rank-1] A matrix $A \ensuremath{\in} \ensuremath{\bbF}^{m\ensuremath{\times}n}$ has rank 1 if and only if there exists $\ensuremath{\bm{\x}} \ensuremath{\in} \ensuremath{\bbF}^m$ and $\ensuremath{\bm{\y}} \ensuremath{\in} \ensuremath{\bbF}^n$ such that
\[
A = \ensuremath{\bm{\x}} \ensuremath{\bm{\y}}^\ensuremath{\top}.
\]
\end{proposition}
\textbf{Proof} This follows immediately as if $A = \ensuremath{\bm{\x}} \ensuremath{\bm{\y}}^\ensuremath{\top}$ then all columns are multiples of $\ensuremath{\bm{\x}}$. On the other hand, if $A$ has rank-1 there exists a nonzero column $\ensuremath{\bm{\x}} := \ensuremath{\bm{\a}}_j$ that all other columns are multiples of. \ensuremath{\QED}

\subsection{LU factorisation}
Gaussian elimination  can be interpreted as an LU factorisation. Write a matrix $A \ensuremath{\in} \ensuremath{\bbF}^{n \ensuremath{\times} n}$ as follows:
\[
A =  \begin{bmatrix} \ensuremath{\alpha}_1 & \ensuremath{\bm{\w}}_1^\ensuremath{\top} \\ \ensuremath{\bm{\v}}_1 & K_1 \end{bmatrix}
\]
where $\ensuremath{\alpha}_1 = a_{11}$, $\ensuremath{\bm{\v}}_1 = A[2:n, 1]$ and $\ensuremath{\bm{\w}}_1 = A[1, 2:n]$ (that is, $\ensuremath{\bm{\v}}_1 \ensuremath{\in} \ensuremath{\bbF}^{n-1}$ is a vector whose entries are the 2nd through last row of the first column of $A$ whilst $\ensuremath{\bm{\w}}_1 \ensuremath{\in} \ensuremath{\bbF}^{n-1}$ is a vector containing the 2nd through last entries in the last column of $A$). Gaussian elimination consists of taking the first row, dividing by $\ensuremath{\alpha}_1$ and subtracting from all other rows. That is equivalent to multiplying by a lower triangular matrix:
\[
\begin{bmatrix}
1 \\
-\ensuremath{\bm{\v}}_1/\ensuremath{\alpha}_1 & I \end{bmatrix} A = \begin{bmatrix} \ensuremath{\alpha}_1 & \ensuremath{\bm{\w}}_1^\ensuremath{\top} \\  & K_1 -\ensuremath{\bm{\v}}_1\ensuremath{\bm{\w}}_1^\ensuremath{\top} /\ensuremath{\alpha}_1 \end{bmatrix}
\]
where $A_2 := K_1 -\ensuremath{\bm{\v}}_1\ensuremath{\bm{\w}}_1^\ensuremath{\top} /\ensuremath{\alpha}_1$  happens to be a rank-1 perturbation of $K_1$. We can write this another way:
\[
A = \underbrace{\begin{bmatrix}
1 \\
\ensuremath{\bm{\v}}_1/\ensuremath{\alpha}_1 & I \end{bmatrix}}_{L_1}  \begin{bmatrix} \ensuremath{\alpha}_1 & \ensuremath{\bm{\w}}_1^\ensuremath{\top} \\  & A_2 \end{bmatrix}
\]
Now assume we continue this process and manage to deduce an LU factorisation $A_2 = L_2 U_2$. Then
\[
A = L_1 \begin{bmatrix} \ensuremath{\alpha}_1 & \ensuremath{\bm{\w}}_1^\ensuremath{\top} \\  & L_2U_2 \end{bmatrix}
= \underbrace{L_1 \begin{bmatrix}
1 \\
 & L_2 \end{bmatrix}}_L  \underbrace{\begin{bmatrix} \ensuremath{\alpha}_1 & \ensuremath{\bm{\w}}_1^\ensuremath{\top} \\  & U_2 \end{bmatrix}}_U
\]
Note we can multiply through to find
\[
L = \begin{bmatrix}
1 \\
\ensuremath{\bm{\v}}_1/\ensuremath{\alpha}_1 & L_2 \end{bmatrix}.
\]
Noting that if $A \ensuremath{\in} \ensuremath{\bbF}^{1 \ensuremath{\times} 1}$ then it has a trivial LU factorisation we can use the above construction to proceed recursively until we arrive at the trivial case.

\begin{example}[LU by-hand] Consider the matrix
\[
A = \begin{bmatrix} 1 & 1 & 1 \\
                    2 & 4 & 8 \\
                    1 & 4 & 9
                    \end{bmatrix} = \underbrace{\begin{bmatrix} 1  \\
                    2 & 1 &  \\
                    1 &  & 1
                    \end{bmatrix}}_{L_1} \begin{bmatrix} 1 & 1 & 1 \\
                    0 & 2 & 6 \\
                    0 & 3 & 8
                    \end{bmatrix}
\]
In more detail, for $\ensuremath{\alpha}_1 := a_{11} = 1$, $\ensuremath{\bm{\v}}_1 := A[2:3,1] = \vectt[2,1]$, $\ensuremath{\bm{\w}}_1 = A[1,2:3] = \vectt[1,1]$ and
\[
K_1 := A[2:3,2:3] = \begin{bmatrix} 4 & 8 \\ 4 & 9 \end{bmatrix}
\]
we have
\[
A_2 := K_1 -\ensuremath{\bm{\v}}_1\ensuremath{\bm{\w}}_1^\ensuremath{\top} /\ensuremath{\alpha}_1 = \begin{bmatrix} 4 & 8 \\ 4 & 9 \end{bmatrix} - \begin{bmatrix} 2 & 2 \\ 1 & 1 \end{bmatrix} = \begin{bmatrix} 2 & 6 \\ 3 & 8 \end{bmatrix}.
\]
We then repeat the process and determine (with $\ensuremath{\alpha}_2 := A_2[1,1] = 2$, $\ensuremath{\bm{\v}}_2 := A_2[2:2,1] = [3]$, $\ensuremath{\bm{\w}}_2 := A_2[1,2:2] = [6]$ and $K_2 := A_2[2:2,2:2] = [8]$):
\[
A_2 =  \begin{bmatrix}2 & 6 \\ 3 & 8 \end{bmatrix} =
\underbrace{\begin{bmatrix}
1 \\
3/2 & 1
\end{bmatrix}}_{L_2} \begin{bmatrix} 2 & 6 \\
            & -1 \end{bmatrix}
\]
The last \ensuremath{\ldq}matrix" is 1 x 1 so we get the trivial decomposition:
\[
A_3 := K_2 - \ensuremath{\bm{\v}}_2 \ensuremath{\bm{\w}}_2^\ensuremath{\top} /\ensuremath{\alpha}_2 =  [-1] = \underbrace{[1]}_{L_3} [-1]
\]
Putting everything together and placing the $j$-th column of $L_j$ inside the $j$-th column of $L$ we have
\[
A = \underbrace{\begin{bmatrix} 1  \\
                    2 & 1 &  \\
                    1 & 3/2 & 1
                    \end{bmatrix}}_{L} \underbrace{\begin{bmatrix} 1 & 1 & 1 \\
                     & 2 & 6 \\
                     &  & -1
                    \end{bmatrix}}_U
\]
\end{example}

\subsection{PLU factorisation}
We learned in first year linear algebra that if a diagonal entry is zero when doing Gaussian elimination one has to \emph{row pivot}. For stability, in implementation one may wish to pivot even if the diagonal entry is nonzero: swap the largest in magnitude entry for the entry on the diagonal turns out to be significantly more stable than standard LU.

This is equivalent to a PLU decomposition. Here we use a \emph{permutation matrix}, whose action on a vector permutes its entries, as discussed in the appendix. That is, consider a permutation which we identify with a vector ${\mathbf \ensuremath{\sigma}} = [\ensuremath{\sigma}_1,\ensuremath{\ldots},\ensuremath{\sigma}_n]$ containing the integers $1,\ensuremath{\ldots},n$ exactly once. The permutation operator represents the action of permuting the entries in a vector:
\[
P_\ensuremath{\sigma}(\ensuremath{\bm{\v}}) := \ensuremath{\bm{\v}}[{\mathbf \ensuremath{\sigma}}] = \Vectt[v_{\ensuremath{\sigma}_1},\ensuremath{\vdots},v_{\ensuremath{\sigma}_n}]
\]
This is a linear operator, and hence we can identify it with a \emph{permutation matrix} $P_\ensuremath{\sigma} \ensuremath{\in} \ensuremath{\bbR}^{n \ensuremath{\times} n}$ (more precisely the entries  of $P_\ensuremath{\sigma}$ are either 1 or 0). Importantly, products of permutation matrices are also permutation matrices and permutation matrices are orthogonal, that is, $P_\ensuremath{\sigma}^{-1} = P_\ensuremath{\sigma}^\top$.

\begin{theorem}[PLU] A matrix $A \ensuremath{\in} \ensuremath{\bbC}^{n \ensuremath{\times} n}$ is invertible if and only if it has a PLU decomposition:
\[
A = P^\ensuremath{\top} L U
\]
where the diagonal of $L$ are all equal to 1 and the diagonal of $U$ are all non-zero, and $P$ is a permutation matrix.

\end{theorem}
\textbf{Proof}

If we have a PLU decomposition of this form then $L$ and $U$ are invertible and hence the inverse is simply $A^{-1} = U^{-1} L^{-1} P$. Hence we consider the orther direction.

If $A \ensuremath{\in} \ensuremath{\bbC}^{1 \ensuremath{\times} 1}$ we trivially have an LU decomposition $A = [1] * [a_{11}]$ as all $1 \ensuremath{\times} 1$ matrices are triangular. We now proceed by induction: assume all invertible matrices of lower dimension have a PLU factorisation. As $A$ is invertible not all entries in the first column are zero. Therefore there exists a permutation $P_1$ so that $\ensuremath{\alpha} := (P_1 A)[1,1] \ensuremath{\neq} 0$. Hence we write
\[
P_1 A = \begin{bmatrix} \ensuremath{\alpha} & \ensuremath{\bm{\w}}^\ensuremath{\top} \\
                        \ensuremath{\bm{\v}} & K
                        \end{bmatrix} = \underbrace{\begin{bmatrix}
1 \\
\ensuremath{\bm{\v}}/\ensuremath{\alpha} & I \end{bmatrix}}_{L_1}  \begin{bmatrix} \ensuremath{\alpha} & \ensuremath{\bm{\w}}^\ensuremath{\top} \\  & K - \ensuremath{\bm{\v}} \ensuremath{\bm{\w}}^\ensuremath{\top}/\ensuremath{\alpha} \end{bmatrix}
\]
We deduce that $A_2 := K - \ensuremath{\bm{\v}} \ensuremath{\bm{\w}}^\ensuremath{\top}/\ensuremath{\alpha}$ is invertible because $A$ and $L_1$ are invertible (Exercise).

By assumption we can write $A_2 = P_2^\ensuremath{\top} L_2 U_2$. Thus we have:
\begin{align*}
\underbrace{\begin{bmatrix} 1 \\
            & P_2 \end{bmatrix} P_1}_P A &= \begin{bmatrix} 1 \\
            & P_2 \end{bmatrix}  \begin{bmatrix} \ensuremath{\alpha} & \ensuremath{\bm{\w}}^\ensuremath{\top} \\
                        \ensuremath{\bm{\v}} & A_2
                        \end{bmatrix}  =
            \begin{bmatrix} 1 \\ & P_2 \end{bmatrix} L_1  \begin{bmatrix} \ensuremath{\alpha} & \ensuremath{\bm{\w}}^\ensuremath{\top} \\  & P_2^\ensuremath{\top} L_2  U_2 \end{bmatrix} \\
            &= \begin{bmatrix}
1 \\
P_2 \ensuremath{\bm{\v}}/\ensuremath{\alpha} & P_2 \end{bmatrix} \begin{bmatrix} 1 &  \\  &  P_2^\ensuremath{\top} L_2  \end{bmatrix}  \begin{bmatrix} \ensuremath{\alpha} & \ensuremath{\bm{\w}}^\ensuremath{\top} \\  &  U_2 \end{bmatrix} \\
&= \underbrace{\begin{bmatrix}
1 \\
P_2 \ensuremath{\bm{\v}}/\ensuremath{\alpha} & L_2  \end{bmatrix}}_L \underbrace{\begin{bmatrix} \ensuremath{\alpha} & \ensuremath{\bm{\w}}^\ensuremath{\top} \\  &  U_2 \end{bmatrix}}_U. \\
\end{align*}
\ensuremath{\QED}

We don't discuss the practical implementation of this factorisation (though an algorithm is hidden in the above proof). We also note that for stability one uses the permutation that always puts the largest in magnitude entry in the top row. In the lab we explore the practical usage of this factorisation.

\subsection{Cholesky factorisation}
A \emph{Cholesky factorisation} is a form of Gaussian elimination (without pivoting) that exploits symmetry in the problem, resulting in a substantial speedup. It is only applicable for \emph{symmetric positive definite} (SPD) matrices, or rather, the algorithm for computing it succeeds if and only if the matrix is SPD. In other words, it gives an algorithmic way to prove whether or not a matrix is SPD.

\begin{definition}[positive definite] A square matrix $A \ensuremath{\in} \ensuremath{\bbR}^{n \ensuremath{\times} n}$ is \emph{positive definite} if for all $\ensuremath{\bm{\x}} \ensuremath{\in} \ensuremath{\bbR}^n, x \ensuremath{\neq} 0$ we have
\[
\ensuremath{\bm{\x}}^\ensuremath{\top} A \ensuremath{\bm{\x}} > 0
\]
\end{definition}

First we establish some basic properties of positive definite matrices:

\begin{proposition}[conj. pos. def.] If  $A \ensuremath{\in} \ensuremath{\bbR}^{n \ensuremath{\times} n}$ is positive definite and $V \ensuremath{\in} \ensuremath{\bbR}^{n \ensuremath{\times} n}$ is non-singular then
\[
V^\ensuremath{\top} A V
\]
is positive definite. \end{proposition}
\textbf{Proof}

For all  $\ensuremath{\bm{\x}} \ensuremath{\in} \ensuremath{\bbR}^n, \ensuremath{\bm{\x}} \ensuremath{\neq} 0$, define $\ensuremath{\bm{\y}} = V \ensuremath{\bm{\x}} \ensuremath{\neq} 0$ (since $V$ is non-singular). Thus we have
\[
\ensuremath{\bm{\x}}^\ensuremath{\top} V^\ensuremath{\top} A V \ensuremath{\bm{\x}} = \ensuremath{\bm{\y}}^\ensuremath{\top} A \ensuremath{\bm{\y}} > 0.
\]
\ensuremath{\QED}

\begin{proposition}[diag positivity] If $A \ensuremath{\in} \ensuremath{\bbR}^{n \ensuremath{\times} n}$ is positive definite then its diagonal entries are positive: $a_{kk} > 0$. \end{proposition}
\textbf{Proof}
\[
a_{kk} = \ensuremath{\bm{\e}}_k^\ensuremath{\top} A \ensuremath{\bm{\e}}_k > 0.
\]
\ensuremath{\QED}

\begin{lemma}[subslice pos. def.] If $A \ensuremath{\in} \ensuremath{\bbR}^{n \ensuremath{\times} n}$ is positive definite and $\ensuremath{\bm{\k}} = [k_1,\ensuremath{\ldots},k_m]^\ensuremath{\top} \ensuremath{\in} \{1,\ensuremath{\ldots},n\}^m$ is a vector of $m$ integers where any integer appears only once,  then $A[\ensuremath{\bm{\k}},\ensuremath{\bm{\k}}] \ensuremath{\in} \ensuremath{\bbR}^{m \ensuremath{\times} m}$ is also positive definite. \end{lemma}
\textbf{Proof} For all $\ensuremath{\bm{\x}} \ensuremath{\in} \ensuremath{\bbR}^m, \ensuremath{\bm{\x}} \ensuremath{\neq} 0$, consider $\ensuremath{\bm{\y}} \ensuremath{\in} \ensuremath{\bbR}^n$ such that $y_{k_j} = x_j$ and zero otherwise. Then we have
\[
\ensuremath{\bm{\x}}^\ensuremath{\top} A[\ensuremath{\bm{\k}},\ensuremath{\bm{\k}}] \ensuremath{\bm{\x}} = \ensuremath{\sum}_{\ensuremath{\ell}=1}^m \ensuremath{\sum}_{j=1}^m x_\ensuremath{\ell} x_j a_{k_\ensuremath{\ell},k_j} = \ensuremath{\sum}_{\ensuremath{\ell}=1}^m \ensuremath{\sum}_{j=1}^m y_{k_\ensuremath{\ell}} y_{k_j} a_{k_\ensuremath{\ell},k_j}  = \ensuremath{\sum}_{\ensuremath{\ell}=1}^n \ensuremath{\sum}_{j=1}^n y_\ensuremath{\ell} y_j a_{\ensuremath{\ell},j} = \ensuremath{\bm{\y}}^\ensuremath{\top} A \ensuremath{\bm{\y}} > 0.
\]
\ensuremath{\QED}

Here is the key result:

\begin{theorem}[Cholesky and SPD] A matrix $A$ is symmetric positive definite if and only if it has a Cholesky factorisation
\[
A = L L^\ensuremath{\top}
\]
where $L$ is lower triangular with positive diagonal entries.

\end{theorem}
\textbf{Proof} If $A$ has a Cholesky factorisation it is symmetric ($A^\ensuremath{\top} = (L L^\ensuremath{\top})^\ensuremath{\top} = A$) and for $\ensuremath{\bm{\x}} \ensuremath{\neq} 0$ we have
\[
\ensuremath{\bm{\x}}^\ensuremath{\top} A \ensuremath{\bm{\x}} = (L^\ensuremath{\top}\ensuremath{\bm{\x}})^\ensuremath{\top} L^\ensuremath{\top} \ensuremath{\bm{\x}} = \|L^\ensuremath{\top}\ensuremath{\bm{\x}}\|^2 > 0
\]
where we use the fact that $L$ is non-singular.

For the other direction we will prove it by induction, with the $1 \ensuremath{\times} 1$ case being trivial. Assume all lower dimensional symmetric positive definite matrices have Cholesky decompositions. Write
\[
A = \begin{bmatrix} \ensuremath{\alpha} & \ensuremath{\bm{\v}}^\ensuremath{\top} \\
                    \ensuremath{\bm{\v}}   & K
                    \end{bmatrix} = \underbrace{\begin{bmatrix} \sqrt{\ensuremath{\alpha}} \\
                                    {\ensuremath{\bm{\v}} \over \sqrt{\ensuremath{\alpha}}} & I \end{bmatrix}}_{L_1}
                                    \begin{bmatrix} 1  \\ & K - {\ensuremath{\bm{\v}} \ensuremath{\bm{\v}}^\ensuremath{\top} \over \ensuremath{\alpha}} \end{bmatrix}
                                    \underbrace{\begin{bmatrix} \sqrt{\ensuremath{\alpha}} & {\ensuremath{\bm{\v}}^\ensuremath{\top} \over \sqrt{\ensuremath{\alpha}}} \\
                                     & I \end{bmatrix}}_{L_1^\ensuremath{\top}}.
\]
Note that $A_2 := K - {\ensuremath{\bm{\v}} \ensuremath{\bm{\v}}^\ensuremath{\top} \over \ensuremath{\alpha}}$ is a subslice of $L_1^{-1} A L_1^{-\ensuremath{\top}}$, hence by combining the previous propositions is itself SPD. Thus we can write
\[
A_2 = K - {\ensuremath{\bm{\v}} \ensuremath{\bm{\v}}^\ensuremath{\top} \over \ensuremath{\alpha}} = L_2 L_2^\ensuremath{\top}
\]
and hence $A = L L^\ensuremath{\top}$ for
\[
L= L_1 \begin{bmatrix}1 \\ & L_2 \end{bmatrix} = \begin{bmatrix} \sqrt{\ensuremath{\alpha}} \\ {\ensuremath{\bm{\v}} \over \sqrt{\ensuremath{\alpha}}} & L_2 \end{bmatrix}
\]
satisfies $A = L L^\ensuremath{\top}$. \ensuremath{\QED}

\begin{example}[Cholesky by hand] Consider the matrix
\[
A = \begin{bmatrix}
2 &1 &1 &1 \\
1 & 2 & 1 & 1 \\
1 & 1 & 2 & 1 \\
1 & 1 & 1 & 2
\end{bmatrix}
\]
Then $\ensuremath{\alpha}_1 = 2$, $\ensuremath{\bm{\v}}_1 = [1,1,1]$, and
\[
A_2 = \begin{bmatrix}
2 &1 &1 \\
1 & 2 & 1 \\
1 & 1 & 2
\end{bmatrix} - {1 \over 2} \begin{bmatrix} 1 \\ 1 \\ 1 \end{bmatrix} \begin{bmatrix} 1 & 1 & 1 \end{bmatrix}
={1 \over 2} \begin{bmatrix}
3 & 1 & 1 \\
1 & 3 & 1 \\
1 & 1 & 3
\end{bmatrix}.
\]
Continuing, we have $\ensuremath{\alpha}_2 = 3/2$, $\ensuremath{\bm{\v}}_2 = [1/2,1/2]$, and
\[
A_3 = {1 \over 2} \left( \begin{bmatrix}
3 & 1 \\ 1 & 3
\end{bmatrix} - {1 \over 3} \begin{bmatrix} 1 \\ 1  \end{bmatrix} \begin{bmatrix} 1 & 1  \end{bmatrix}
\right)
= {1 \over 3} \begin{bmatrix} 4 & 1 \\ 1 & 4 \end{bmatrix}
\]
Next, $\ensuremath{\alpha}_3 = 4/3$, $\ensuremath{\bm{\v}}_3 = [1]$, and
\[
A_4 = [4/3 - 3/4 * (1/3)^2] = [5/4]
\]
i.e. $\ensuremath{\alpha}_4 = 5/4$.

Thus we get
\[
L= \begin{bmatrix}
\sqrt{\ensuremath{\alpha}_1} \\
{\ensuremath{\bm{\v}}_1[1] \over \sqrt{\ensuremath{\alpha}_1}} & \sqrt{\ensuremath{\alpha}_2} \\
{\ensuremath{\bm{\v}}_1[2] \over \sqrt{\ensuremath{\alpha}_1}} & {\ensuremath{\bm{\v}}_2[1] \over \sqrt{\ensuremath{\alpha}_2}}  & \sqrt{\ensuremath{\alpha}_3} \\
{\ensuremath{\bm{\v}}_1[3] \over \sqrt{\ensuremath{\alpha}_1}} & {\ensuremath{\bm{\v}}_2[2] \over \sqrt{\ensuremath{\alpha}_2}}  & {\ensuremath{\bm{\v}}_3[1] \over \sqrt{\ensuremath{\alpha}_3}}  & \sqrt{\ensuremath{\alpha}_4}
\end{bmatrix}
 = \begin{bmatrix} \sqrt{2} \\ {1 \over \sqrt{2}} & \sqrt{3 \over 2} \\
{1 \over \sqrt{2}} & {1 \over \sqrt 6} & {2 \over \sqrt{3}} \\
{1 \over \sqrt{2}} & {1 \over \sqrt 6} & {1 \over \sqrt{12}} & {\sqrt{5} \over 2}
\end{bmatrix}
\]
\end{example}





\section{Polynomial Interpolation and Regression}
In this section we switch tracks and begin to consider numerical linear algebra related to rectangular matrices and least squares systems, which we motivate with an application to polynomial regression. \emph{Polynomial interpolation} is the process of finding a polynomial that equals data at a precise set of points. A more robust scheme is \emph{polynomial regression} where we use more data than the degrees of freedom in the polynomial. We therefore determine the polynomial using \emph{least squares}: find the polynomial whose samples at the points are as close as possible to the data, as measured in the $2$-norm. This least squares problem is done numerically which will be discussed in the next few sections.

\subsection{Polynomial interpolation}
Our prelimary goal is given a set of points $x_j$ and data $f_j$, usually samples of a function $f_j = f(x_j)$, find a polynomial that interpolates the data at the points:

\begin{definition}[interpolatory polynomial] Given \emph{distinct} points $\ensuremath{\bm{\x}} = \vectt[x_1,\ensuremath{\ldots},x_n] \ensuremath{\in} \ensuremath{\bbF}^n$ and \emph{data} $\ensuremath{\bm{\f}} = \vectt[f_1,\ensuremath{\ldots},f_n] \ensuremath{\in} \ensuremath{\bbF}^n$, a degree $n-1$ \emph{interpolatory polynomial} $p(x)$ satisfies
\[
p(x_j) = f_j
\]
\end{definition}

The easiest way to solve this problem is to invert the Vandermonde system:

\begin{definition}[Vandermonde] The \emph{Vandermonde matrix} associated with $\ensuremath{\bm{\x}} \ensuremath{\in} \ensuremath{\bbF}^m$ is the matrix
\[
V_{\ensuremath{\bm{\x}},n} := \begin{bmatrix} 1 & x_1 & \ensuremath{\cdots} & x_1^{n-1} \\
                    \ensuremath{\vdots} & \ensuremath{\vdots} & \ensuremath{\ddots} & \ensuremath{\vdots} \\
                    1 & x_m & \ensuremath{\cdots} & x_m^{n-1}
                    \end{bmatrix} \ensuremath{\in} \ensuremath{\bbF}^{m \ensuremath{\times} n}.
\]
When it is clear from context we omit the subscripts $\ensuremath{\bm{\x}},n$. \end{definition}

Writing the coefficients of a polynomial
\[
p(x) = \ensuremath{\sum}_{k=0}^{n-1} c_k x^k
\]
as a vector  $\ensuremath{\bm{\c}} = \vectt[c_0,\ensuremath{\ldots},c_{n-1}] \ensuremath{\in} \ensuremath{\bbF}^n$, we note that $V$ encodes the linear map from coefficients to values at a grid, that is,
\[
V\ensuremath{\bm{\c}} = \Vectt[c_0 + c_1 x_1 + \ensuremath{\cdots} + c_{n-1} x_1^{n-1}, \ensuremath{\vdots}, c_0 + c_1 x_m + \ensuremath{\cdots} + c_{n-1} x_m^{n-1}] = \Vectt[p(x_1),\ensuremath{\vdots},p(x_m)].
\]
In the square case (where $m=n$), the coefficients of an interpolatory polynomial are given by $\ensuremath{\bm{\c}} = V^{-1} \ensuremath{\bm{\f}}$, so that
\[
\Vectt[p(x_1),\ensuremath{\vdots},p(x_n)] = V \ensuremath{\bm{\c}} = V V^{-1} \ensuremath{\bm{\f}} = \Vectt[f_1,\ensuremath{\vdots},f_n].
\]
This inversion is justified by the following:

\begin{proposition}[interpolatory polynomial uniqueness] Interpolatory polynomials are unique and therefore square Vandermonde matrices are invertible.

\end{proposition}
\textbf{Proof} Suppose $p$ and $\pt$ are both interpolatory polynomials of the same function. Then $p(x) - \pt(x)$ vanishes at $n$ distinct points $x_j$. By the fundamental theorem of algebra it must be zero, i.e., $p = \pt$.

For the second part, if $V \ensuremath{\bm{\c}} = 0$ for $\ensuremath{\bm{\c}} = \vectt[c_0,\ensuremath{\ldots},c_{n-1}] \ensuremath{\in} \ensuremath{\bbF}^n$ then for $q(x) = c_0 + \ensuremath{\cdots} + c_{n-1} x^{n-1}$ we have
\[
q(x_j) = \ensuremath{\bm{\e}}_j^\ensuremath{\top} V \ensuremath{\bm{\c}} = 0
\]
hence $q$ vanishes at $n$ distinct points and is therefore 0, i.e., $\ensuremath{\bm{\c}} = 0$.

\ensuremath{\QED}

We can invert square Vandermonde matrix numerically in $O(n^3)$ operations using the PLU factorisation. But it turns out we can also construct the interpolatory polynomial directly, and evaluate the polynomial in only $O(n^2)$ operations. We will use the following polynomials which equal $1$ at one grid point and zero at the others:

\begin{definition}[Lagrange basis polynomial] The \emph{Lagrange basis polynomial} is defined as
\[
\ensuremath{\ell}_k(x) := \ensuremath{\prod}_{j \ensuremath{\neq} k} {x-x_j \over x_k - x_j} =  {(x-x_1) \ensuremath{\cdots}(x-x_{k-1})(x-x_{k+1}) \ensuremath{\cdots} (x-x_n) \over (x_k - x_1) \ensuremath{\cdots} (x_k - x_{k-1}) (x_k - x_{k+1}) \ensuremath{\cdots} (x_k - x_n)}
\]
\end{definition}

Plugging in the grid points verifies that: $\ensuremath{\ell}_k(x_j) = \ensuremath{\delta}_{kj}$.

We can use these to construct the interpolatory polynomial:

\begin{theorem}[Lagrange interpolation] The unique interpolation polynomial is:
\[
p(x) = f_1 \ensuremath{\ell}_1(x) + \ensuremath{\cdots} + f_n \ensuremath{\ell}_n(x)
\]
\end{theorem}
\textbf{Proof} Note that
\[
p(x_j) = \ensuremath{\sum}_{j=1}^n f_j \ensuremath{\ell}_k(x_j) = f_j.
\]
\ensuremath{\QED}

\begin{example}[interpolating an exponential] We can interpolate $\exp(x)$ at the points $0,1,2$. That is, our data is $\ensuremath{\bm{\f}} = \vectt[{\rm e}, {\rm e},{\rm e}^2]$ and the interpolatory polynomial is
\begin{align*}
p(x) &= \ensuremath{\ell}_1(x) + {\rm e} \ensuremath{\ell}_2(x) + {\rm e}^2 \ensuremath{\ell}_3(x) =
{(x - 1) (x-2) \over (-1)(-2)} + {\rm e} {x (x-2) \over (-1)} +
{\rm e}^2 {x (x-1) \over 2} \\
&= (1/2 - {\rm e} +{\rm e}^2/2)x^2 + (-3/2 + 2 {\rm e}  - {\rm e}^2 /2) x + 1
\end{align*}
\end{example}

\textbf{Remark} Interpolating at evenly spaced points is a really \emph{bad} idea: interpolation is inheritely ill-conditioned. The labs will explore this issue experimentally. Another serious issue is that monomials are a horrible basis for interpolation. This is intuitive: when $n$ is large $x^n$ is basically zero near the origin and hence $x_j^n$ numerically lose linear independence, that is, on a computer they appear to be linearly dependent (up to rounding errors). We will discuss alternative bases in Part IV.

\subsection{Polynomial regression}
In many settings interpolation is not an accurate or appropriate tool. Data is often on an evenly spaced grid in which case (as seen in the labs) interpolation breaks down catastrophically. Or the data is noisy and one ends up over resolving: approximating the noise rather than the signal. A simple solution is \emph{polynomial regression} use more sample points than than the degrees of freedom in the polynomial. The special case of an affine polynomial is called \emph{linear regression}.

More precisely, for $\ensuremath{\bm{\x}} \ensuremath{\in} \ensuremath{\bbF}^m$ and for $n < m$ we want to find a degree $n-1$ polynomial
\[
p(x) = \ensuremath{\sum}_{k=0}^{n-1} c_k x^k
\]
such that
\[
\Vectt[p(x_1), \ensuremath{\vdots}, p(x_m)] \ensuremath{\approx} \underbrace{\Vectt[f_1,\ensuremath{\vdots},f_m]}_{\ensuremath{\bm{\f}}}.
\]
Mapping between coefficients $\ensuremath{\bm{\c}} \ensuremath{\in} \ensuremath{\bbF}^n$ to polynomial values on a grid can be accomplished  via rectangular Vandermonde matrices. In particular, our goal is to choose $\ensuremath{\bm{\c}} \ensuremath{\in} \ensuremath{\bbF}^n$ so that
\[
V \ensuremath{\bm{\c}}  = \Vectt[p(x_1), \ensuremath{\vdots}, p(x_m)] \ensuremath{\approx} \ensuremath{\bm{\f}}.
\]
We do so by solving the \emph{least squares} system: given $V \ensuremath{\in} \ensuremath{\bbF}^{m \ensuremath{\times} n}$ and $\ensuremath{\bm{\f}} \ensuremath{\in} \ensuremath{\bbF}^m$ we want to find $\ensuremath{\bm{\c}} \ensuremath{\in} \ensuremath{\bbF}^n$ such that
\[
\| V \ensuremath{\bm{\c}} - \ensuremath{\bm{\f}} \|
\]
is minimal. Note interpolation is a special case where this norm is precisely zero (which is indeed minimal), but in general this norm may be rather large.   We will discuss the numerical solution of least squares problems in the next few sections.

\textbf{Remark} Using regression instead of interpolation can overcome the issues with evenly spaced grids. However, the monomial basis is still very problematic.





\section{Orthogonal and Unitary Matrices}
To solve least squares problems, we will  factorise a matrix $A$ as $A = QR$ where $R$ is a \emph{right-triangular matrix} and $Q$ is either an \emph{orthogonal} or \emph{unitary} matrix.

\begin{definition}[orthogonal/unitary matrix] A square real matrix is \emph{orthogonal} if its inverse is its transpose:
\[
O(n) = \{Q \ensuremath{\in} \ensuremath{\bbR}^{n \ensuremath{\times} n} : Q^\ensuremath{\top}Q = I \}
\]
A square complex matrix is \emph{unitary} if its inverse is its adjoint:
\[
U(n) = \{Q \ensuremath{\in} \ensuremath{\bbC}^{n \ensuremath{\times} n} : Q^\ensuremath{\star}Q = I \}.
\]
Here the adjoint is the same as the conjugate-transpose: $Q^\ensuremath{\star} := \bar Q^\ensuremath{\top}$.  \end{definition}

Note that $O(n) \ensuremath{\subset} U(n)$ as for real matrices $Q^\ensuremath{\star} = Q^\ensuremath{\top}$. Because in either case $Q^{-1} = Q^\ensuremath{\star}$ we also have $Q Q^\ensuremath{\star} = I$ (which for real matrices is $Q Q^\ensuremath{\top} = I$). These matrices are particularly important for numerical linear algebra for a number of reasons (we'll explore these properties in the problem sheets):

\begin{itemize}
\item[1. ] They are norm-preserving: for any vector $\ensuremath{\bm{\x}} \ensuremath{\in} \ensuremath{\bbC}^n$ and $Q \ensuremath{\in} U(n)$    we have $\|Q \ensuremath{\bm{\x}} \| = \| \ensuremath{\bm{\x}}\|$ where $\| \ensuremath{\bm{\x}} \|^2 := \ensuremath{\sum}_{k=1}^n x_k^2$ (i.e. the 2-norm).


\item[2. ] All eigenvalues have absolute value equal to $1$.


\item[3. ] For $Q \ensuremath{\in} O(n)$,  $\det Q = \ensuremath{\pm}1$.


\item[4. ] They are trivially invertible (just take the adjoint).


\item[5. ] They are generally \ensuremath{\ldq}stable": errors due to rounding when multiplying a vector by $Q$ are controlled.


\item[6. ] They are \emph{normal matrices}: they commute with their adjoint ($Q Q^\ensuremath{\star} = Q Q^\ensuremath{\star}$). 


\item[7. ] Both $O(n)$ and $U(n)$ are groups, in particular, they are closed under multiplication.

\end{itemize}
On a computer there are multiple ways of representing orthogonal/unitary matrices, and it is almost never to store a dense matrix, that is, we do not want to store all the entries. In the appendices we have seen permutation matrices, which are a special type of orthogonal matrices where we can store only the order the entries are permuted as a vector. 

More generally, we will use the group structure: represent general orthogonal/unitary matrices as products of simpler elements of the group. In partular we will use two building blocks:

\begin{itemize}
\item[1. ] \emph{Rotations}: Rotations are equivalent to special orthogonal matrices $SO(2)$  and correspond to rotations in 2D.


\item[2. ] \emph{Reflections}:  Reflections are elements of $U(n)$ that are defined in terms of a single unit vector $\ensuremath{\bm{\v}} \ensuremath{\in} \ensuremath{\bbC}^n$ which is reflected.

\end{itemize}
We remark a related concept to orthogonal/unitary matrices are rectangular matrices with orthonormal columns, e.g.
\[
U = [\ensuremath{\bm{\u}}_1 | \ensuremath{\cdots} | \ensuremath{\bm{\u}}_n] \ensuremath{\in} \ensuremath{\bbC}^{m \ensuremath{\times} n}
\]
where $m \ensuremath{\geq} n$ such that $U^\ensuremath{\star} U =  I_n$ (the $n \ensuremath{\times} n$ identity matrix). In this case we must have $UU^\ensuremath{\star} \ensuremath{\neq} I_m$ as the rank of $U$ is $n < m$. 

\subsection{Rotations}
We begin with a general definition:

\begin{definition}[Special Orthogonal and Rotations] \emph{Special Orthogonal Matrices} are
\[
SO(n) := \{Q \ensuremath{\in} O(n) | \det Q = 1 \}
\]
And (simple) \emph{rotations} are $SO(2)$. \end{definition}

In what follows we use the following for writing the angle of a vector:

\begin{definition}[two-arg arctan] The two-argument arctan function gives the angle \texttt{\ensuremath{\theta}} through the point $[a,b]^\ensuremath{\top}$, i.e., 
\[
\sqrt{a^2 + b^2} \begin{bmatrix} \cos \ensuremath{\theta} \\ \sin \ensuremath{\theta} \end{bmatrix} =  \begin{bmatrix} a \\ b \end{bmatrix}.
\]
It can be defined in terms of the standard arctan as follows:
\[
{\rm atan}(b,a) := \begin{cases} {\rm atan}{b \over a} & a > 0 \\
                            {\rm atan}{b \over a} + \ensuremath{\pi} & a < 0\hbox{ and }b >0 \\
                            {\rm atan}{b \over a} - \ensuremath{\pi} & a < 0\hbox{ and }b < 0 \\
                            \ensuremath{\pi}/2 & a = 0\hbox{ and }b >0 \\
                            -\ensuremath{\pi}/2 & a = 0\hbox{ and }b < 0 
                            \end{cases}
\]
\end{definition}

We show $SO(2)$ are exactly equivalent to standard rotations:

\begin{proposition}[simple rotation] A 2\ensuremath{\times}2 \emph{rotation matrix} through angle $\ensuremath{\theta}$ is
\[
Q_\ensuremath{\theta} := \begin{bmatrix} \cos \ensuremath{\theta} & -\sin \ensuremath{\theta} \cr \sin \ensuremath{\theta} & \cos \ensuremath{\theta} \end{bmatrix}.
\]
We have $Q \ensuremath{\in} SO(2)$ if and only if $Q = Q_\ensuremath{\theta}$ for some $\ensuremath{\theta} \ensuremath{\in} \ensuremath{\bbR}$.

\end{proposition}
\textbf{Proof}

We will write $c = \cos \ensuremath{\theta}$ and $s = \sin \ensuremath{\theta}$. Then we have
\[
Q_\ensuremath{\theta}^\ensuremath{\top}Q_\ensuremath{\theta} = \begin{pmatrix} c & s \\ -s & c \end{pmatrix} \begin{pmatrix} c & -s \\ s & c \end{pmatrix} = 
\begin{pmatrix} c^2 + s^2 & 0 \\ 0 & c^2 + s^2 \end{pmatrix} = I
\]
and $\det Q_\ensuremath{\theta} = c^2 + s^2 = 1$ hence $Q_\ensuremath{\theta} \ensuremath{\in} SO(2)$. 

Now suppose $Q = [\ensuremath{\bm{\q}}_1, \ensuremath{\bm{\q}}_2] \ensuremath{\in} SO(2)$ where we know its columns have norm 1, i.e. $\|\ensuremath{\bm{\q}}_k\| = 1$, and are orthogonal. Write $\ensuremath{\bm{\q}}_1 = [c,s]$ where we know $c = \cos \ensuremath{\theta}$ and $s = \sin \ensuremath{\theta}$ for $\ensuremath{\theta} = {\rm atan}(s, c)$.  Since $\ensuremath{\bm{\q}}_1\cdot \ensuremath{\bm{\q}}_2 = 0$ we can deduce $\ensuremath{\bm{\q}}_2 = \ensuremath{\pm} [-s,c]$. The sign is positive as $\det Q = \ensuremath{\pm}(c^2 + s^2) = \ensuremath{\pm}1$.

\ensuremath{\QED}

We can rotate an arbitrary vector in $\ensuremath{\bbR}^2$ to the unit axis using rotations, which are useful in linear algebra decompositions. Interestingly it only requires basic algebraic functions (no trigonometric functions):

\begin{proposition}[rotation of a vector]  The matrix
\[
Q = {1 \over \sqrt{a^2 + b^2}}
\begin{bmatrix}
 a & b \cr -b & a
\end{bmatrix}
\]
is a rotation matrix ($Q \ensuremath{\in} SO(2)$) satisfying
\[
Q \begin{bmatrix} a \\ b \end{bmatrix} = \sqrt{a^2 + b^2} \begin{bmatrix} 1 \\ 0 \end{bmatrix}
\]
\end{proposition}
\textbf{Proof} 

The last equation is trivial so the only question is that it is a rotation matrix. This follows immediately:
\[
Q^\ensuremath{\top} Q = {1 \over a^2 + b^2}  \begin{bmatrix}
 a^2 + b^2 & 0 \cr 0 & a^2 + b^2
\end{bmatrix} = I
\]
and $\det Q = 1$.

\ensuremath{\QED}

\begin{example}[rotating a vector] Consider the vector
\[
\ensuremath{\bm{\x}} = \Vectt[-1,-\sqrt{3}].
\]
We can use the proposition above to deduce the rotation matrix that rotates this vector to the positive real axis is:
\[
{1 \over \sqrt{1+3}} \begin{bmatrix} -1 & -\sqrt{3} \\ \sqrt{3} & -1 \end{bmatrix} = 
{1 \over 2} \begin{bmatrix} -1 & -\sqrt{3} \\ \sqrt{3} & -1 \end{bmatrix}.
\]
Alternatively, we could determine the matrix by computing the angle of the vector via:
\[
\ensuremath{\theta} =  {\rm atan}(-\sqrt{3}, -1) = {\rm atan}(\sqrt{3}) - \ensuremath{\pi} = -{2\ensuremath{\pi} \over 3}.
\]
We thus compute:
\[
Q_{-\ensuremath{\theta}} = \begin{bmatrix}
\cos(2\ensuremath{\pi}/3) & -\sin(2\ensuremath{\pi}/3) \\
\sin(2\ensuremath{\pi}/3) & \cos(2\ensuremath{\pi}/3)
\end{bmatrix} = {1 \over 2} \begin{bmatrix} -1 & -\sqrt{3} \\ \sqrt{3} & -1 \end{bmatrix}.
\]
\end{example}

More generally, we can consider rotations that operate on two entries of a vector at a time. This will be explored in the problem sheet/lab.

\subsection{Reflections}
In addition to rotations, another type of orthogonal/unitary matrix are reflections. These are specified by a single vector which is reflected, with everything orthogonal to the vector left fixed. 

\begin{definition}[reflection matrix]  Given a unit vector $\ensuremath{\bm{\v}} \ensuremath{\in} \ensuremath{\bbC}^n$ (satisfying $\|\ensuremath{\bm{\v}}\|=1$), define the corresponding \emph{reflection matrix} as:
\[
Q_{\ensuremath{\bm{\v}}} := I - 2 \ensuremath{\bm{\v}} \ensuremath{\bm{\v}}^\ensuremath{\star}
\]
\end{definition}

These are indeed reflections in the direction of $\ensuremath{\bm{\v}}$. We can show this as follows:

\begin{proposition}[Householder properties] $Q_{\ensuremath{\bm{\v}}}$ satisfies:

\begin{itemize}
\item[1. ] Symmetry: $Q_{\ensuremath{\bm{\v}}} = Q_{\ensuremath{\bm{\v}}}^\ensuremath{\star}$


\item[2. ] Orthogonality: $Q_{\ensuremath{\bm{\v}}} \ensuremath{\in} U(n)$


\item[3. ] The vector $\ensuremath{\bm{\v}}$ is an eigenvector of $Q_{\ensuremath{\bm{\v}}}$ with eigenvalue $-1$


\item[4. ] For the dimension $n-1$ space $W := \{\ensuremath{\bm{\w}} : \ensuremath{\bm{\w}}^\ensuremath{\star} \ensuremath{\bm{\v}} = 0 \}$, all vectors $\ensuremath{\bm{\w}} \ensuremath{\in} W$ satisfy $Q_{\ensuremath{\bm{\v}}}\ensuremath{\bm{\w}} = \ensuremath{\bm{\w}}$.


\item[5. ] Not a rotation: $\det Q_{\ensuremath{\bm{\v}}} = -1$

\end{itemize}
\end{proposition}
\textbf{Proof}

Property 1 follows immediately. Property 2 follows from
\[
Q_{\ensuremath{\bm{\v}}}^\ensuremath{\star} Q_{\ensuremath{\bm{\v}}} = Q_{\ensuremath{\bm{\v}}}^2 = I - 4 \ensuremath{\bm{\v}} \ensuremath{\bm{\v}}^\ensuremath{\star} + 4 \ensuremath{\bm{\v}} \ensuremath{\bm{\v}}^\ensuremath{\star} \ensuremath{\bm{\v}} \ensuremath{\bm{\v}}^\ensuremath{\star} = I.
\]
Property 3 follows since
\[
Q_{\ensuremath{\bm{\v}}} \ensuremath{\bm{\v}} = \ensuremath{\bm{\v}} - 2\ensuremath{\bm{\v}} (\ensuremath{\bm{\v}}^\ensuremath{\star}\ensuremath{\bm{\v}}) = -\ensuremath{\bm{\v}}.
\]
Property 4 follows from:
\[
Q_{\ensuremath{\bm{\v}}} \ensuremath{\bm{\w}} = \ensuremath{\bm{\w}} - 2 \ensuremath{\bm{\v}} (\ensuremath{\bm{\w}}^\ensuremath{\star} \ensuremath{\bm{\v}}) =  \ensuremath{\bm{\w}}
\]
Property 5 then follows: Property 4 tells us that $1$ is an eigenvalue with multiplicity $n-1$. Since $-1$ is an eigenvalue with multiplicity 1,  the determinant, which is product of the eigenvalues, is $-1$.

\ensuremath{\QED}

\begin{example}[reflection through 2-vector] Consider reflection through $\ensuremath{\bm{\x}} = [1,2]^\ensuremath{\top}$.  We first need to normalise $\ensuremath{\bm{\x}}$:
\[
\ensuremath{\bm{\v}} = {\ensuremath{\bm{\x}} \over \|\ensuremath{\bm{\x}}\|} = \begin{bmatrix} {1 \over \sqrt{5}} \\ {2 \over \sqrt{5}} \end{bmatrix}
\]
The reflection matrix is:
\[
Q_{\ensuremath{\bm{\v}}} = I - 2 \ensuremath{\bm{\v}} \ensuremath{\bm{\v}}^\ensuremath{\top} = \begin{bmatrix}1 \\ & 1 \end{bmatrix} - {2 \over 5} \begin{bmatrix} 1 & 2 \\ 2 & 4 \end{bmatrix}
 =  {1 \over 5} \begin{bmatrix} 3 & -4 \\ -4 & -3 \end{bmatrix}
\]
Indeed it is symmetric, and orthogonal. It sends $\ensuremath{\bm{\x}}$ to $-\ensuremath{\bm{\x}}$:
\[
Q_{\ensuremath{\bm{\v}}} \ensuremath{\bm{\x}} = {1 \over 5} \begin{bmatrix}3 - 8 \\ -4 - 6 \end{bmatrix} = -\ensuremath{\bm{\x}}
\]
Any vector orthogonal to $\ensuremath{\bm{\x}}$, like $\ensuremath{\bm{\y}} = [-2,1]^\ensuremath{\top}$, is left fixed:
\[
Q_{\ensuremath{\bm{\v}}} \ensuremath{\bm{\y}} = {1 \over 5} \begin{bmatrix}-6 -4 \\ 8 - 3 \end{bmatrix} = \ensuremath{\bm{\y}}
\]
\end{example}

Note that \emph{building} the matrix $Q_{\ensuremath{\bm{\v}}}$ will be expensive ($O(n^2)$ operations), but we can \emph{apply} $Q_{\ensuremath{\bm{\v}}}$ to a vector in $O(n)$ operations using the expression:
\[
Q_{\ensuremath{\bm{\v}}} \ensuremath{\bm{\x}} = \ensuremath{\bm{\x}} - 2 \ensuremath{\bm{\v}} (\ensuremath{\bm{\v}}^\ensuremath{\star} \ensuremath{\bm{\x}}) = \ensuremath{\bm{\x}} - 2 \ensuremath{\bm{\v}} (\ensuremath{\bm{\v}} \ensuremath{\cdot} \ensuremath{\bm{\x}}).
\]
\subsubsection{Householder reflections}
Just as rotations can be used to rotate vectors to be aligned with coordinate axis, so can reflections, but in this case it works for vectors in $\ensuremath{\bbC}^n$, not just $\ensuremath{\bbR}^2$. We begin with the real case:

\begin{definition}[Householder reflection, real case] For a given vector $\ensuremath{\bm{\x}} \ensuremath{\in} \ensuremath{\bbR}^n$, define the Householder reflection
\[
Q_{\ensuremath{\bm{\x}}}^{\ensuremath{\pm},\rm H} := Q_{\ensuremath{\bm{\w}}}
\]
for $\ensuremath{\bm{\y}} = \ensuremath{\mp} \|\ensuremath{\bm{\x}}\| \ensuremath{\bm{\e}}_1 + \ensuremath{\bm{\x}}$ and $\ensuremath{\bm{\w}} = {\ensuremath{\bm{\y}} \over \|\ensuremath{\bm{\y}}\|}$. The default choice in sign is:
\[
Q_{\ensuremath{\bm{\x}}}^{\rm H} := Q_{\ensuremath{\bm{\x}}}^{-\hbox{sign}(x_1),\rm H}.
\]
\end{definition}

\begin{lemma}[Householder reflection maps to axis] For $\ensuremath{\bm{\x}} \ensuremath{\in} \ensuremath{\bbR}^n$,
\[
Q_{\ensuremath{\bm{\x}}}^{\ensuremath{\pm},\rm H} \ensuremath{\bm{\x}} = \ensuremath{\pm}\|\ensuremath{\bm{\x}}\| \ensuremath{\bm{\e}}_1
\]
\end{lemma}
\textbf{Proof} Note that
\begin{align*}
\| \ensuremath{\bm{\y}} \|^2 &= 2\|\ensuremath{\bm{\x}}\|^2 \ensuremath{\mp} 2 \|\ensuremath{\bm{\x}}\| x_1, \\
\ensuremath{\bm{\y}}^\ensuremath{\top} \ensuremath{\bm{\x}} &= \|\ensuremath{\bm{\x}}\|^2 \ensuremath{\mp}  \|\ensuremath{\bm{\x}}\| x_1
\end{align*}
where $x_1 = \ensuremath{\bm{\e}}_1^\ensuremath{\top} \ensuremath{\bm{\x}}$. Therefore:
\[
Q_{\ensuremath{\bm{\x}}}^{\ensuremath{\pm},\rm H} \ensuremath{\bm{\x}}  =  (I - 2 \ensuremath{\bm{\w}} \ensuremath{\bm{\w}}^\ensuremath{\top}) \ensuremath{\bm{\x}} = \ensuremath{\bm{\x}} - 2 {\ensuremath{\bm{\y}}  \|\ensuremath{\bm{\x}}\|  \over \|\ensuremath{\bm{\y}}\|^2} (\|\ensuremath{\bm{\x}}\|\ensuremath{\mp}x_1) = \ensuremath{\bm{\x}} - \ensuremath{\bm{\y}} =  \ensuremath{\pm}\|\ensuremath{\bm{\x}}\| \ensuremath{\bm{\e}}_1.
\]
\ensuremath{\QED}

\textbf{Remark} Why do we choose the the opposite sign of $x_1$ for the default reflection? For stability, but we won't discuss this in more detail.

We can extend this definition for complexes:

\begin{definition}[Householder reflection, complex case] For a given vector $\ensuremath{\bm{\x}} \ensuremath{\in} \ensuremath{\bbC}^n$, define the Householder reflection as
\[
Q_{\ensuremath{\bm{\x}}}^{\rm H} := Q_{\ensuremath{\bm{\w}}}
\]
for $\ensuremath{\bm{\y}} = {\rm csign}(x_1) \|\ensuremath{\bm{\x}}\| \ensuremath{\bm{\e}}_1 + \ensuremath{\bm{\x}}$ and $\ensuremath{\bm{\w}} = {\ensuremath{\bm{\y}} \over \|\ensuremath{\bm{\y}}\|}$, for ${\rm csign}(z) = {\rm e}^{{\rm i} \arg z}$.  \end{definition}

\begin{lemma}[Householder reflection maps to axis, complex case] For $\ensuremath{\bm{\x}} \ensuremath{\in} \ensuremath{\bbC}^n$,
\[
Q_{\ensuremath{\bm{\x}}}^{\rm H} \ensuremath{\bm{\x}} = -{\rm csign}(x_1) \|\ensuremath{\bm{\x}}\| \ensuremath{\bm{\e}}_1
\]
\end{lemma}
\textbf{Proof} Denote $\ensuremath{\alpha} := {\rm csign}(x_1)$.  Note that $\baralpha x_1 = {\rm e}^{-{\rm i} \arg x_1} x_1 = |x_1|$.  Now we have
\begin{align*}
\| \ensuremath{\bm{\y}} \|^2 &= (\ensuremath{\alpha} \|\ensuremath{\bm{\x}}\| \ensuremath{\bm{\e}}_1 + \ensuremath{\bm{\x}})^\ensuremath{\star}(\ensuremath{\alpha} \|\ensuremath{\bm{\x}}\| \ensuremath{\bm{\e}}_1 + \ensuremath{\bm{\x}}) = |\ensuremath{\alpha}|\| \ensuremath{\bm{\x}} \|^2 + \| \ensuremath{\bm{\x}} \|  \ensuremath{\alpha} \bar x_1 + \baralpha x_1 \| \ensuremath{\bm{\x}} \| + \| \ensuremath{\bm{\x}} \|^2 \\
&= 2\| \ensuremath{\bm{\x}} \|^2 + 2|x_1| \| \ensuremath{\bm{\x}} \| \\
\ensuremath{\bm{\y}}^\ensuremath{\star} \ensuremath{\bm{\x}} &= \baralpha x_1 \| \ensuremath{\bm{\x}} \| + \|\ensuremath{\bm{\x}} \|^2 = \|\ensuremath{\bm{\x}} \|^2 + |x_1| \| \ensuremath{\bm{\x}} \|
\end{align*}
Therefore:
\[
Q_{\ensuremath{\bm{\x}}}^{\rm H} \ensuremath{\bm{\x}}  =  (I - 2 \ensuremath{\bm{\w}} \ensuremath{\bm{\w}}^\ensuremath{\star}) \ensuremath{\bm{\x}} = \ensuremath{\bm{\x}} - 2 {\ensuremath{\bm{\y}}    \over \|\ensuremath{\bm{\y}}\|^2} (\|\ensuremath{\bm{\x}} \|^2 + |x_1| \|\ensuremath{\bm{\x}} \|) = \ensuremath{\bm{\x}} - \ensuremath{\bm{\y}} =  -\ensuremath{\alpha} \|\ensuremath{\bm{\x}}\| \ensuremath{\bm{\e}}_1.
\]
\ensuremath{\QED}





\section{QR Factorisation}
Let $A \ensuremath{\in} \ensuremath{\bbC}^{m \ensuremath{\times} n}$ be a rectangular or square matrix such that $m \ensuremath{\geq} n$ (i.e. more rows then columns). In this chapter we consider two closely related factorisations:

\begin{definition}[QR factorisation] The \emph{QR factorisation} is
\[
A = Q R = \underbrace{\begin{bmatrix} \ensuremath{\bm{\q}}_1 | \ensuremath{\cdots} | \ensuremath{\bm{\q}}_m \end{bmatrix}}_{Q \ensuremath{\in} U(m)} \underbrace{\begin{bmatrix} \ensuremath{\times} & \ensuremath{\cdots} & \ensuremath{\times} \\ & \ensuremath{\ddots} & \ensuremath{\vdots} \\ && \ensuremath{\times} \\ &&0 \\ &&\ensuremath{\vdots} \\ && 0 \end{bmatrix}}_{R \ensuremath{\in} \ensuremath{\bbC}^{m \ensuremath{\times} n}}
\]
where $Q$ is unitary (i.e., $Q \ensuremath{\in} U(m)$, satisfying $Q^\ensuremath{\star}Q = I$, with columns $\ensuremath{\bm{\q}}_j \ensuremath{\in} \ensuremath{\bbC}^m$) and $R$ is \emph{right triangular}, which means it  is only nonzero on or to the right of the diagonal ($r_{kj} = 0$ if $k > j$). \end{definition}

\begin{definition}[Reduced QR factorisation] The \emph{reduced QR factorisation}
\[
A = \hat Q \hat R = \underbrace{\begin{bmatrix} \ensuremath{\bm{\q}}_1 | \ensuremath{\cdots} | \ensuremath{\bm{\q}}_n \end{bmatrix}}_{ \hat Q \ensuremath{\in} \ensuremath{\bbC}^{m \ensuremath{\times} n}} \underbrace{\begin{bmatrix} \ensuremath{\times} & \ensuremath{\cdots} & \ensuremath{\times} \\ & \ensuremath{\ddots} & \ensuremath{\vdots} \\ && \ensuremath{\times}  \end{bmatrix}}_{\hat R \ensuremath{\in} \ensuremath{\bbC}^{n \ensuremath{\times} n}}
\]
where $\hat Q$ has orthonormal columns ($\hat Q^\ensuremath{\star} \hat Q = I$, $\ensuremath{\bm{\q}}_j \ensuremath{\in} \ensuremath{\bbC}^m$) and $\hat R$ is upper triangular. \end{definition}

Note for a square matrix the reduced QR factorisation is equivalent to the QR factorisation, in which case $R$ is \emph{upper triangular}. The importance of these factorisation for square matrices is that their component pieces are easy to invert:
\[
A = QR \qquad \ensuremath{\Rightarrow} \qquad A^{-1}\ensuremath{\bm{\b}} = R^{-1} Q^\ensuremath{\top} \ensuremath{\bm{\b}}
\]
and we saw previously that triangular and orthogonal matrices are easy to invert when applied to a vector $\ensuremath{\bm{\b}}$.

For rectangular matrices we will see that the QR factorisation leads to efficient solutions to the \emph{least squares problem}: find $\ensuremath{\bm{\x}}$ that minimizes the 2-norm $\| A \ensuremath{\bm{\x}} - \ensuremath{\bm{\b}} \|.$ Note in the rectangular case the QR factorisation contains within it the reduced QR factorisation:
\[
A = QR = \begin{bmatrix} \hat Q | \ensuremath{\bm{\q}}_{n+1} | \ensuremath{\cdots} | \ensuremath{\bm{\q}}_m \end{bmatrix} \begin{bmatrix} \hat R \\  \ensuremath{\bm{\zero}}_{m-n \ensuremath{\times} n} \end{bmatrix} = \hat Q \hat R.
\]
In this chapter we discuss the following:

\begin{itemize}
\item[1. ] Reduced QR and Gram\ensuremath{\endash}Schmidt: We discuss computation of the Reduced QR factorisation using Gram\ensuremath{\endash}Schmidt.


\item[2. ] Householder reflections and QR: We discuss computing the  QR factorisation using Householder reflections. This is a more accurate approach

\end{itemize}
for computing QR factorisations.

\begin{itemize}
\item[3. ] QR and least squares: We discuss the QR factorisation and its usage in solving least squares problems.

\end{itemize}
\subsection{Reduced QR and Gram\ensuremath{\endash}Schmidt}
How do we compute the QR factorisation? We begin with a method you may have seen before in another guise. Write
\[
A = \begin{bmatrix} \ensuremath{\bm{\a}}_1 | \ensuremath{\cdots} | \ensuremath{\bm{\a}}_n \end{bmatrix}
\]
where $\ensuremath{\bm{\a}}_k \ensuremath{\in}  \ensuremath{\bbC}^m$ and assume they are linearly independent ($A$ has full column rank).

\begin{proposition}[Column spaces match] Suppose $A = \hat Q  \hat R$ where $\hat Q = [\ensuremath{\bm{\q}}_1|\ensuremath{\ldots}|\ensuremath{\bm{\q}}_n]$ has orthonormal columns and $\hat R$ is upper-triangular, and $A$ has full rank. Then the first $j$ columns of $\hat Q$ span the same space as the first $j$ columns of $A$:
\[
\hbox{span}(\ensuremath{\bm{\a}}_1,\ensuremath{\ldots},\ensuremath{\bm{\a}}_j) = \hbox{span}(\ensuremath{\bm{\q}}_1,\ensuremath{\ldots},\ensuremath{\bm{\q}}_j).
\]
\end{proposition}
\textbf{Proof}

Because $A$ has full rank we know $\hat R$ is invertible, i.e. its diagonal entries do not vanish: $r_{jj} \ensuremath{\neq} 0$. If $\ensuremath{\bm{\v}} \ensuremath{\in} \hbox{span}(\ensuremath{\bm{\a}}_1,\ensuremath{\ldots},\ensuremath{\bm{\a}}_j)$ we have for $\ensuremath{\bm{\c}} \ensuremath{\in} \ensuremath{\bbC}^j$
\[
\ensuremath{\bm{\v}} = \begin{bmatrix} \ensuremath{\bm{\a}}_1 | \ensuremath{\cdots} | \ensuremath{\bm{\a}}_j \end{bmatrix} \ensuremath{\bm{\c}} = 
\begin{bmatrix} \ensuremath{\bm{\q}}_1 | \ensuremath{\cdots} | \ensuremath{\bm{\q}}_j \end{bmatrix}  \hat R[1:j,1:j] \ensuremath{\bm{\c}} \ensuremath{\in} \hbox{span}(\ensuremath{\bm{\q}}_1,\ensuremath{\ldots},\ensuremath{\bm{\q}}_j)
\]
while if $\ensuremath{\bm{\w}} \ensuremath{\in} \hbox{span}(\ensuremath{\bm{\q}}_1,\ensuremath{\ldots},\ensuremath{\bm{\q}}_j)$ we have for $\vc d \ensuremath{\in} \ensuremath{\bbR}^j$
\[
\ensuremath{\bm{\w}} = \begin{bmatrix} \ensuremath{\bm{\q}}_1 | \ensuremath{\cdots} | \ensuremath{\bm{\q}}_j \end{bmatrix} \vc d  =  \begin{bmatrix} \ensuremath{\bm{\a}}_1 | \ensuremath{\cdots} | \ensuremath{\bm{\a}}_j \end{bmatrix} \hat R[1:j,1:j]^{-1} \vc d \ensuremath{\in}  \hbox{span}(\ensuremath{\bm{\a}}_1,\ensuremath{\ldots},\ensuremath{\bm{\a}}_j).
\]
\ensuremath{\QED}

It is possible to find $\hat Q$ and $\hat R$ the  using the \emph{Gram\ensuremath{\endash}Schmidt algorithm}. We construct it column-by-column. For $j = 1, 2, \ensuremath{\ldots}, n$ define
\begin{align*}
\ensuremath{\bm{\v}}_j &:= \ensuremath{\bm{\a}}_j - \ensuremath{\sum}_{k=1}^{j-1} \underbrace{\ensuremath{\bm{\q}}_k^\ensuremath{\star} \ensuremath{\bm{\a}}_j}_{r_{kj}} \ensuremath{\bm{\q}}_k \\
r_{jj} &:= {\|\ensuremath{\bm{\v}}_j\|} \\
\ensuremath{\bm{\q}}_j &:= {\ensuremath{\bm{\v}}_j \over r_{jj}}
\end{align*}
\textbf{Theorem (Gram\ensuremath{\endash}Schmidt and reduced QR)} Define $\ensuremath{\bm{\q}}_j$ and $r_{kj}$ as above (with $r_{kj} = 0$ if $k > j$). Then a reduced QR factorisation is given by:
\[
A = \underbrace{\begin{bmatrix} \ensuremath{\bm{\q}}_1 | \ensuremath{\cdots} | \ensuremath{\bm{\q}}_n \end{bmatrix}}_{ \hat Q \ensuremath{\in} \ensuremath{\bbC}^{m \ensuremath{\times} n}} \underbrace{\begin{bmatrix} r_{11} & \ensuremath{\cdots} & r_{1n} \\ & \ensuremath{\ddots} & \ensuremath{\vdots} \\ && r_{nn}  \end{bmatrix}}_{\hat R \ensuremath{\in} \ensuremath{\bbC}^{n \ensuremath{\times} n}}
\]
\textbf{Proof}

We first show that $\hat Q$ has orthonormal columns. Assume that $\ensuremath{\bm{\q}}_\ensuremath{\ell}^\ensuremath{\star} \ensuremath{\bm{\q}}_k = \ensuremath{\delta}_{\ensuremath{\ell}k}$ for $k,\ensuremath{\ell} < j$.  For $\ensuremath{\ell} < j$ we then have
\[
\ensuremath{\bm{\q}}_\ensuremath{\ell}^\ensuremath{\star} \ensuremath{\bm{\v}}_j = \ensuremath{\bm{\q}}_\ensuremath{\ell}^\ensuremath{\star} \ensuremath{\bm{\a}}_j - \ensuremath{\sum}_{k=1}^{j-1}  \ensuremath{\bm{\q}}_\ensuremath{\ell}^\ensuremath{\star}\ensuremath{\bm{\q}}_k \ensuremath{\bm{\q}}_k^\ensuremath{\star} \ensuremath{\bm{\a}}_j = 0
\]
hence $\ensuremath{\bm{\q}}_\ensuremath{\ell}^\ensuremath{\star} \ensuremath{\bm{\q}}_j = 0$ and indeed $\hat Q$ has orthonormal columns. Further: from the definition of $\ensuremath{\bm{\v}}_j$ we find
\[
\ensuremath{\bm{\a}}_j = \ensuremath{\bm{\v}}_j + \ensuremath{\sum}_{k=1}^{j-1} r_{kj} \ensuremath{\bm{\q}}_k = \ensuremath{\sum}_{k=1}^j r_{kj} \ensuremath{\bm{\q}}_k  = \hat Q \hat R \ensuremath{\bm{\e}}_j
\]
\ensuremath{\QED}

\subsection{Householder reflections and QR}
As an alternative, we will consider using Householder reflections to introduce zeros below the diagonal. Thus, if Gram\ensuremath{\endash}Schmidt is a process of \emph{triangular orthogonalisation} (using triangular matrices to orthogonalise), Householder reflections is a process of \emph{orthogonal triangularisation}  (using orthogonal matrices to triangularise).

Consider multiplication by the Householder reflection corresponding to the first column, that is, for
\[
Q_1 := Q_{\ensuremath{\bm{\a}}_1}^{\rm H},
\]
consider
\[
Q_1 A = \begin{bmatrix} \ensuremath{\times} & \ensuremath{\times} & \ensuremath{\cdots} & \ensuremath{\times} \\
& \ensuremath{\times} & \ensuremath{\cdots} & \ensuremath{\times} \\
                    & \ensuremath{\vdots} & \ensuremath{\ddots} & \ensuremath{\vdots} \\
                    & \ensuremath{\times} & \ensuremath{\cdots} & \ensuremath{\times} \end{bmatrix} = 
\begin{bmatrix}  \ensuremath{\alpha}_1 & \ensuremath{\bm{\w}}_1^\ensuremath{\top} \\ 
& A_2   \end{bmatrix}
\]
where 
\[
\ensuremath{\alpha}_1 := -{\rm csign}(a_{11})  \|\ensuremath{\bm{\a}}_1\|, \ensuremath{\bm{\w}}_1 = (Q_1 A)[1, 2:n]  \qquad \hbox{and} \qquad A_2 = (Q_1 A)[2:m, 2:n],
\]
where as before ${\rm csign}(z) :=  {\rm e}^{{\rm i} \arg z}$. That is, we have made the first column triangular. In terms of an algorithm, we then introduce zeros into the first column of $A_2$, leaving an $A_3$, and so-on. But we can wrap this iterative algorithm into a simple proof by induction, reminisicent of our proof for the Cholesky factorisation:

\begin{theorem}[QR]  Every matrix $A \ensuremath{\in} \ensuremath{\bbC}^{m \ensuremath{\times} n}$ has a QR factorisation:
\[
A = QR
\]
where $Q \ensuremath{\in} U(m)$ and $R \ensuremath{\in} \ensuremath{\bbC}^{m \ensuremath{\times} n}$ is right triangular.

\end{theorem}
\textbf{Proof}

First assume $m \ensuremath{\geq} n$. If $A = [\ensuremath{\bm{\a}}_1] \ensuremath{\in} \ensuremath{\bbC}^{m \ensuremath{\times} 1}$ then we have for the Householder reflection $Q_1 = Q_{\ensuremath{\bm{\a}}_1}^{\rm H}$
\[
Q_1 A = \ensuremath{\alpha} \ensuremath{\bm{\e}}_1
\]
which is right triangular, where $\ensuremath{\alpha} = -{\rm csign}(a_{11}) \|\ensuremath{\bm{\a}}_1\|$.  In other words 
\[
A = \underbrace{Q_1}_Q \underbrace{\ensuremath{\alpha} \ensuremath{\bm{\e}}_1}_R.
\]
For $n > 1$, assume every matrix with less columns than $n$ has a QR factorisation. For $A = [\ensuremath{\bm{\a}}_1|\ensuremath{\ldots}|\ensuremath{\bm{\a}}_n] \ensuremath{\in} \ensuremath{\bbC}^{m \ensuremath{\times} n}$, let $Q_1 = Q_{\ensuremath{\bm{\a}}_1}^{\rm H}$ so that
\[
Q_1 A =  \begin{bmatrix} \ensuremath{\alpha} & \ensuremath{\bm{\w}}^\ensuremath{\top} \\ & A_2 \end{bmatrix}.
\]
By assumption $A_2 = Q_2 R_2$. Thus we have (recalling that $Q_1^{-1} = Q_1^\ensuremath{\star} = Q_1$):
\begin{align*}
A = Q_1 \begin{bmatrix} \ensuremath{\alpha} & \ensuremath{\bm{\w}}^\ensuremath{\top} \\ & Q_2 R_2 \end{bmatrix} \\
=\underbrace{Q_1 \begin{bmatrix} 1 \\ & Q_2 \end{bmatrix}}_Q  \underbrace{\begin{bmatrix} \ensuremath{\alpha} & \ensuremath{\bm{\w}}^\ensuremath{\top} \\ &  R_2 \end{bmatrix}}_R.
\end{align*}
If $m < n$, i.e., $A$ has more columns then rows, write 
\[
A = \begin{bmatrix} \At & B \end{bmatrix}
\]
where $\At \ensuremath{\in} \ensuremath{\bbC}^{m \ensuremath{\times} m}$. From above we know we can write $\At = Q \Rt$. We thus have
\[
A = Q \underbrace{\begin{bmatrix} \Rt & Q^\ensuremath{\star} B \end{bmatrix}}_R
\]
where $R$ is right triangular.

\ensuremath{\QED}

\begin{example}[QR by hand, non-examinable] We will now do an example by hand. Consider the $4 \ensuremath{\times} 3$ matrix
\[
A = \begin{bmatrix} 
2 & 3 & 0 \\ 
0 & 0 & 1 \\
-2 & -3 & 0 \\
-1 & -3 & -3
\end{bmatrix}
\]
For the first column we have
\[
\ensuremath{\bm{\y}}_1 := [-1,0,-2,-1]
\]
where $\| \ensuremath{\bm{\y}}_1 \|^2 = 6$. Hence
\[
Q_1 := I - {1 \over 3} \begin{bmatrix} -1 \\ 0 \\ -2 \\ -1 \end{bmatrix} \begin{bmatrix} -1 & 0 & -2 & -1 \end{bmatrix} =
 {1 \over 3} \begin{bmatrix}
2 & 0 & -2 & -1 \\
0 & 3 & 0 & 0 \\
-2 & 0 & -1 & -2 \\
-1 & 0 & -2 &  2
\end{bmatrix}
\]
so that
\[
Q_1 A = \begin{bmatrix} 3 &  5 & 1 \\
 & 0 & 1 \\
  & 1 & 2 \\
& -1 & -2
\end{bmatrix}
\]
For the second column we have
\[
\ensuremath{\bm{\y}}_2 :=  [-\sqrt{2},1,-1]
\]
where $\| \ensuremath{\bm{\y}}_2 \|^2 = 4$. Thus we have
\[
Q_2 := I - {1 \over 2}
 \begin{bmatrix} -\sqrt{2} \\1 \\ -1
\end{bmatrix} \begin{bmatrix} -\sqrt{2} & 1 & -1 \end{bmatrix}
= \begin{bmatrix}
0 & 1/\sqrt{2} & -1/\sqrt{2} \\
1/\sqrt{2} & 1/2 & 1/2 \\
-1/\sqrt{2} & 1/2 & 1/2
\end{bmatrix}
\]
so that
\[
\tilde Q_2 Q_1 A = \begin{bmatrix} 3 & 5 & 1 \\
 & \sqrt{2} & 2\sqrt{2} \\
  & 0 & 1/\sqrt{2} \\
& 0 & -1/\sqrt{2}
\end{bmatrix}
\]
The final vector is 
\[
\ensuremath{\bm{\y}}_3 := [1/\sqrt{2}-1,-1/\sqrt{2}]
\]
where $\| \ensuremath{\bm{\y}}_3 \|^2 = 2 - 2/\sqrt{2}$. Hence
\[
Q_3 := I - {\sqrt{2} \over \sqrt{2} - 1} \begin{bmatrix}
1/\sqrt{2}-1 \\
-1/\sqrt{2}
\end{bmatrix} \begin{bmatrix}
1/\sqrt{2}-1 &
-1/\sqrt{2}
\end{bmatrix} =
\begin{bmatrix}
\sqrt{2} & -\sqrt{2}\\
-\sqrt{2} & -\sqrt{2}
\end{bmatrix}
\]
so that 
\[
\tilde Q_3 \tilde Q_2 Q_1 A = \begin{bmatrix} 3 & 5 & 1 \\
 & \sqrt{2} & 2\sqrt{2} \\
  & 0 & 1 \\
& 0 & 0
\end{bmatrix} =: R
\]
and
\[
Q := Q_1 \tilde Q_2 \tilde Q_3 =  \begin{bmatrix}
2/3 & -1/(3\sqrt{2}) & 0 & 1/\sqrt{2} \\
0 &  0 & 1 & 0 \\
-2/3 & 1/(3\sqrt{2}) & 0 & 1/\sqrt{2} \\ 
-1/3 & - 4/(3\sqrt{2}) & 0 & 0
\end{bmatrix}.
\]
\subsection{QR and least squares}
We consider rectangular matrices with more rows than columns. Given $A \ensuremath{\in} \ensuremath{\bbC}^{m \ensuremath{\times} n}$ and $\ensuremath{\bm{\b}} \ensuremath{\in} \ensuremath{\bbC}^m$, least squares consists of finding a vector $\ensuremath{\bm{\x}} \ensuremath{\in} \ensuremath{\bbC}^n$ that minimises the 2-norm: $\| A \ensuremath{\bm{\x}} - \ensuremath{\bm{\b}} \|$.

\begin{theorem}[least squares via QR] Suppose $A \ensuremath{\in} \ensuremath{\bbC}^{m \ensuremath{\times} n}$ with $m \ensuremath{\geq} n$ has full rank. Given a QR factorisation $A = Q R$ then
\[
\ensuremath{\bm{\x}} = \hat R^{-1} \hat Q^\ensuremath{\star} \ensuremath{\bm{\b}}
\]
minimises $\| A \ensuremath{\bm{\x}} - \ensuremath{\bm{\b}} \|$. 

\end{theorem}
\textbf{Proof}

The norm-preserving property ($\|Q\ensuremath{\bm{\x}}\| = \|\ensuremath{\bm{\x}}\|$) of unitary matrices tells us
\[
\| A \ensuremath{\bm{\x}} - \ensuremath{\bm{\b}} \| = \| Q R \ensuremath{\bm{\x}} - \ensuremath{\bm{\b}} \| = \| Q (R \ensuremath{\bm{\x}} - Q^\ensuremath{\star} \ensuremath{\bm{\b}}) \| = \| R \ensuremath{\bm{\x}} - Q^\ensuremath{\star} \ensuremath{\bm{\b}} \| = \left \| 
\begin{bmatrix} \hat R \\ \ensuremath{\bm{\zero}}_{m-n \ensuremath{\times} n} \end{bmatrix} \ensuremath{\bm{\x}} - \begin{bmatrix} \hat Q^\ensuremath{\star} \\ \ensuremath{\bm{\q}}_{n+1}^\ensuremath{\star} \\ \ensuremath{\vdots} \\ \ensuremath{\bm{\q}}_m^\ensuremath{\star} \end{bmatrix}     \ensuremath{\bm{\b}} \right \|
\]
Now note that the rows $k > n$ are independent of $\ensuremath{\bm{\x}}$ and are a fixed contribution. Thus to minimise this norm it suffices to drop them and minimise:
\[
\| \hat R \ensuremath{\bm{\x}} - \hat Q^\ensuremath{\star} \ensuremath{\bm{\b}} \|
\]
This norm is minimised if it is zero. Provided the column rank of $A$ is full, $\hat R$ will be invertible.

\end{example}






\chapter{Approximation Theory}

So far, we have seen intuitive numerical methods for computing derivatives, integrals, and solving
differential equations, primarily based on representing functions by their values at a grid of points.
But by using more sophisticated mathematical tools, we can achieve much more accurate and reliable
numerical methods. In particular, we can effectively use Fourier series for computing very accurately with periodic functions,
and orthogonal polynomials for non-periodic functions that are smooth within an interval.
Here we introduce these fundamental tools and explore applications to quadrature (computing integrals) where they
produce incredibly accurate approximations, ones that converge exponentially (or faster) for analytic functions.

\begin{enumerate}
    \item IV.1 Fourier Expansions: we discuss Fourier series and their usage in approximating periodic functions, using the Trapezium rule to compute the Fourier coefficients.
    \item IV.2 Discrete Fourier Transform: The Trapezium rule approximation can be recast as a unitary matrix, known as the Discrete Fourier Transform (DFT). This is used to prove interpolation properties.
    \item IV.3 Orthogonal Polynomials: For non-periodic functions we consider orthogonal polynomials, and discuss their basic properties.
    \item IV.4 Classical Orthogonal Polynomials: For certain weights, orthogonal polynomials are classical and have addition structure that are useful for computations.
    \item IV.5 Gaussian Quadrature: Finally, we revisit the  problem of computing integrals, and see that using orthogonal polynomials we can derive much more accurate methods.
    \end{enumerate}

We stop at integration, but Fourier and orthogonal polynomial expansions also lead to very effective scheme for solving differential equations
and many other applications.


\section{Fourier Expansions}
Fourier series are a powerful tool in wide areas of mathematics, including solving partial differential equations, signal processing, and elsewhere. They are also very useful in computational methods, particularly for problems that have periodicity. Periodicity arises naturally when solving problems in radial coordinates, or when approximating a problem on the real line by a periodic problem with a large period. Fourier series are also related to orthogonal polynomials, which can be used for non-periodic problems.

\subsection{Basics of Fourier series}
The most fundamental basis is (complex) Fourier: we have ${\rm e}^{{\rm i} k \ensuremath{\theta}}$ are orthogonal with respect to the inner product
\[
\ensuremath{\langle}f, g \ensuremath{\rangle} := {1 \over 2\ensuremath{\pi}} \ensuremath{\int}_0^{2\ensuremath{\pi}} \bar f(\ensuremath{\theta}) g(\ensuremath{\theta}) {\rm d}\ensuremath{\theta},
\]
where we conjugate the first argument to be consistent with the vector inner product $\ensuremath{\bm{\x}}^\ensuremath{\star} \ensuremath{\bm{\y}}$. We will use the notation $\ensuremath{\bbT} := [0,2\ensuremath{\pi})$ (typically this has the topology of a circle attached but we do not need to worry about that here). We can (typically) expand functions in this basis:

\begin{definition}[Fourier] A function $f$ has a Fourier expansion if
\[
f(\ensuremath{\theta}) = \ensuremath{\sum}_{k = -\ensuremath{\infty}}^\ensuremath{\infty} \hat f_k {\rm e}^{{\rm i} k \ensuremath{\theta}}
\]
where
\[
\hat f_k := \ensuremath{\langle}{\rm e}^{{\rm i} k \ensuremath{\theta}}, f\ensuremath{\rangle} = {1 \over 2\ensuremath{\pi}} \ensuremath{\int}_0^{2\ensuremath{\pi}}  {\rm e}^{-{\rm i} k \ensuremath{\theta}} f(\ensuremath{\theta}) {\rm d}\ensuremath{\theta}
\]
\end{definition}

A basic observation is if a Fourier expansion has no negative terms it is equivalent to a Taylor series in disguise:

\begin{definition}[Fourier-Taylor] A function $f$ has a Fourier\ensuremath{\endash}Taylor expansion if
\[
f(\ensuremath{\theta}) = \ensuremath{\sum}_{k = 0}^\ensuremath{\infty} \hat f_k {\rm e}^{{\rm i} k \ensuremath{\theta}} = \ensuremath{\sum}_{k = 0}^\ensuremath{\infty} \hat f_k z^k
\]
where $\hat f_k := \ensuremath{\langle}{\rm e}^{{\rm i} k \ensuremath{\theta}}, f\ensuremath{\rangle}$, and $z = {\rm e}^{{\rm i} \ensuremath{\theta}}$. \end{definition}

In numerical analysis we try to build on the analogy with linear algebra as much as possible. Therefore we  can write this this as:
\[
f(\ensuremath{\theta}) = \underbrace{[1 | {\rm e}^{{\rm i}\ensuremath{\theta}} | {\rm e}^{2{\rm i}\ensuremath{\theta}} | \ensuremath{\cdots}]}_{T(\ensuremath{\theta})}
\underbrace{\begin{bmatrix} \hat f_0 \\ \hat f_1 \\ \hat f_2 \\ \ensuremath{\vdots} \end{bmatrix}}_{\vchatf}.
\]
Essentially, expansions in bases are viewed as a way of turning \emph{functions} into (infinite) \emph{vectors}. And (differential) \emph{operators} into \emph{matrices}.

In analysis one typically works with continuous functions and relates results to continuity. In numerical analysis we inheritely have to work with \emph{vectors}, so it is more natural to  focus on the case where the \emph{Fourier coefficients} $\hat f_k$ are \emph{absolutely convergent}:

\begin{definition}[absolute convergent] We write $\vchatf \ensuremath{\in} \ensuremath{\ell}^1$ if it is absolutely convergent, or in otherwords, the $1$-norm of $\vchatf$ is bounded:
\[
\|\vchatf\|_1 := \ensuremath{\sum}_{k=-\ensuremath{\infty}}^\ensuremath{\infty} |\hat f_k| < \ensuremath{\infty}.
\]
\end{definition}

We first state a  basic results (whose proof is beyond the scope of this module):

\begin{theorem}[Fourier series equivalence] If $f, g : \ensuremath{\bbT} \ensuremath{\rightarrow} \ensuremath{\bbC}$ are continuous and $\hat f_k = \hat g_k$ for all $k \ensuremath{\in} \ensuremath{\bbZ}$ then $f = g$.

\end{theorem}
\textbf{Proof} See \href{https://www.cambridge.org/core/books/fourier-analysis/5FD8F0FD69DDB139019655D7F8440D2F}{Körner 2022 (Theorem 2.4)}. \ensuremath{\QED}

This allows us to prove the following:

\begin{theorem}[Absolute converging Fourier series] If $\vchatf \ensuremath{\in} \ensuremath{\ell}^1$ then
\[
f(\ensuremath{\theta}) = \ensuremath{\sum}_{k = -\ensuremath{\infty}}^\ensuremath{\infty} \hat f_k {\rm e}^{{\rm i} k \ensuremath{\theta}},
\]
which converges uniformly. \end{theorem}
\textbf{Proof}

Note that
\[
g_n(\ensuremath{\theta}) := \ensuremath{\sum}_{k = -n}^n \hat f_k {\rm e}^{{\rm i} k \ensuremath{\theta}}
\]
is uniformly-absolutely convergent as $n \ensuremath{\rightarrow} \ensuremath{\infty}$, that is,
\[
\ensuremath{\sum}_{k = -n}^n |\hat f_k {\rm e}^{{\rm i} k \ensuremath{\theta}}| = \ensuremath{\sum}_{k = -n}^n |\hat f_k| \ensuremath{\rightarrow} \|\vchatf\|_1.
\]
This guarantees that $g_n(\ensuremath{\theta})$ converges uniformly to a continuous function $g(\ensuremath{\theta})$. We have for $n > k$, that the $k$-th Fourier coefficient of $g_n(\ensuremath{\theta})$ equals $\hat f_k$. Thus, by the properties of uniform convergence,
\[
\hat f_k = \lim_{n \ensuremath{\rightarrow} \ensuremath{\infty}} \hat f_k =  \lim_{n \ensuremath{\rightarrow} \ensuremath{\infty}} {1 \over 2\ensuremath{\pi}} \ensuremath{\int}_0^{2\ensuremath{\pi}}  {\rm e}^{-{\rm i} k \ensuremath{\theta}} g_n(\ensuremath{\theta}) {\rm d}\ensuremath{\theta} =
 {1 \over 2\ensuremath{\pi}} \ensuremath{\int}_0^{2\ensuremath{\pi}}  {\rm e}^{-{\rm i} k \ensuremath{\theta}} \lim_{n \ensuremath{\rightarrow} \ensuremath{\infty}} g_n(\ensuremath{\theta}) {\rm d}\ensuremath{\theta} = \hat g_k.
\]
Since $f$ and $g$ are continuous and share the same Fourier coefficients, they are equal.

\ensuremath{\QED}

When does a function have absolutely convergent Fourier coefficients? We can deduce it from periodic differentiability of the function:

\begin{lemma}[differentiability and absolutely convergence] If $f : \ensuremath{\bbR} \ensuremath{\rightarrow} \ensuremath{\bbC}$ and $f'$ are periodic  and $f''$ is uniformly bounded, then $\vchatf \ensuremath{\in} \ensuremath{\ell}^1$.

\end{lemma}
\textbf{Proof} Integrate by parts twice using the fact that $f(0) = f(2\ensuremath{\pi})$, $f'(0) = f'(2\ensuremath{\pi})$:
\begin{align*}
2\ensuremath{\pi}\hat f_k &= \ensuremath{\int}_0^{2\ensuremath{\pi}} f(\ensuremath{\theta}) {\rm e}^{-{\rm i} k \ensuremath{\theta}} {\rm d}\ensuremath{\theta} =
[f(\ensuremath{\theta}) {\rm e}^{-{\rm i} k \ensuremath{\theta}}]_0^{2\ensuremath{\pi}} + {1 \over {\rm i} k} \ensuremath{\int}_0^{2\ensuremath{\pi}} f'(\ensuremath{\theta}) {\rm e}^{-{\rm i} k \ensuremath{\theta}} {\rm d}\ensuremath{\theta} \\
&= {1 \over {\rm i} k} [f'(\ensuremath{\theta}) {\rm e}^{-{\rm i} k \ensuremath{\theta}}]_0^{2\ensuremath{\pi}} - {1 \over k^2} \ensuremath{\int}_0^{2\ensuremath{\pi}} f''(\ensuremath{\theta}) {\rm e}^{-{\rm i} k \ensuremath{\theta}} {\rm d}\ensuremath{\theta} \\
&= - {1 \over k^2} \ensuremath{\int}_0^{2\ensuremath{\pi}} f''(\ensuremath{\theta}) {\rm e}^{-{\rm i} k \ensuremath{\theta}} {\rm d}\ensuremath{\theta}.
\end{align*}
Thus uniform boundedness of $f''$ guarantees $|\hat f_k| \ensuremath{\leq} M |k|^{-2}$ for some $M$, and we have
\[
\ensuremath{\sum}_{k = -\ensuremath{\infty}}^\ensuremath{\infty} |\hat f_k| \ensuremath{\leq} |\hat f_0|  + 2M \ensuremath{\sum}_{k = 1}^\ensuremath{\infty} |k|^{-2}  < \ensuremath{\infty}
\]
using the dominant convergence test.

\ensuremath{\QED}

This condition can be weakened to Lipschitz continuity but the proof is  beyond the scope of this module. Of more practical importance is the other direction: the more times differentiable a function the faster the coefficients decay, and thence the faster Fourier expansions converge. In fact, if a function is smooth and 2\ensuremath{\pi}-periodic its Fourier coefficients decay faster than algebraically: they decay like $O(k^{-\ensuremath{\lambda}})$ for any $\ensuremath{\lambda}$. This will be explored in the problem sheet.

\subsection{Trapezium rule and discrete Fourier coefficients}
\begin{definition}[Trapezium Rule] Let $\ensuremath{\theta}_j = 2\ensuremath{\pi}j/n$ for $j = 0,1,\ensuremath{\ldots},n$ denote $n+1$ evenly spaced points over $[0,2\ensuremath{\pi}]$. Recall that the \emph{Trapezium rule} over $[0,2\ensuremath{\pi}]$ is the approximation:
\[
\ensuremath{\int}_0^{2\ensuremath{\pi}} f(\ensuremath{\theta}) {\rm d}\ensuremath{\theta} \ensuremath{\approx} {2 \ensuremath{\pi} \over n} \left[{f(0) \over 2} + \ensuremath{\sum}_{j=1}^{n-1} f(\ensuremath{\theta}_j) + {f(2 \ensuremath{\pi}) \over 2} \right]
\]
But if $f$ is periodic we have $f(0) = f(2\ensuremath{\pi})$ and we get the \emph{periodic Trapezium rule}:
\[
{1 \over 2\ensuremath{\pi}} \ensuremath{\int}_0^{2\ensuremath{\pi}} f(\ensuremath{\theta}) {\rm d}\ensuremath{\theta} \ensuremath{\approx} \underbrace{{1 \over n} \ensuremath{\sum}_{j=0}^{n-1} f(\ensuremath{\theta}_j)}_{\ensuremath{\Sigma}_n[f]}
\]
\end{definition}

We know that ${\rm e}^{{\rm i} k \ensuremath{\theta}}$ are orthogonal with respect to the continuous inner product. The following says that this property is maintained (up to \ensuremath{\ldq}aliasing") when we replace the continuous integral with a trapezium rule approximation:

\begin{lemma}[Discrete orthogonality] We have:
\[
\ensuremath{\sum}_{j=0}^{n-1} {\rm e}^{{\rm i} k \ensuremath{\theta}_j} =
\begin{cases} n & k = \ldots,-2n,-n,0,n,2n,\ldots  \cr
              0 & \hbox{otherwise}
\end{cases}
\]
In other words,
\[
\ensuremath{\Sigma}_n[{\rm e}^{{\rm i} (k-\ensuremath{\ell}) \ensuremath{\theta}}] =
\begin{cases} 1 & k-\ensuremath{\ell} = \ldots,-2n,-n,0,n,2n,\ldots  \cr
              0 & \hbox{otherwise}
\end{cases}.
\]
\end{lemma}
\textbf{Proof}

Consider $\ensuremath{\omega} := {\rm e}^{{\rm i} \ensuremath{\theta}_1} = {\rm e}^{2 \ensuremath{\pi} {\rm i} \over n}$. This is an $n$-th root of unity: $\ensuremath{\omega}^n = 1$. Note that ${\rm e}^{{\rm i} \ensuremath{\theta}_j} ={\rm e}^{2 \ensuremath{\pi} {\rm i} j \over n}= \ensuremath{\omega}^j$.

(Case 1: $k = pn$ for an integer $p$) We have
\[
\ensuremath{\sum}_{j=0}^{n-1} {\rm e}^{{\rm i} k \ensuremath{\theta}_j} = \ensuremath{\sum}_{j=0}^{n-1} \ensuremath{\omega}^{kj} = \ensuremath{\sum}_{j=0}^{n-1} ({\ensuremath{\omega}^{pn}})^j =   \ensuremath{\sum}_{j=0}^{n-1} 1 = n
\]
(Case 2: $k \ensuremath{\neq} pn$ for an integer $p$)  Recall that (via a telescoping sum argument)
\[
\ensuremath{\sum}_{j=0}^{n-1} z^j = {z^n-1 \over z-1}.
\]
Then we have
\[
\ensuremath{\sum}_{j=0}^{n-1} {\rm e}^{{\rm i} k \ensuremath{\theta}_j} = \ensuremath{\sum}_{j=0}^{n-1} (\ensuremath{\omega}^k)^j = {\ensuremath{\omega}^{kn} -1 \over \ensuremath{\omega}^k -1} = 0.
\]
where we use the fact that $k$ is not a multiple of $n$ to guarantee that $\ensuremath{\omega}^k \ensuremath{\neq} 1$.

\ensuremath{\QED}

\subsection{Convergence of Approximate Fourier expansions}
We will now use the Trapezium rule to approximate Fourier coefficients and expansions:

\begin{definition}[Discrete Fourier coefficients] Define the Trapezium rule approximation to the Fourier coefficients by:
\[
\hat f_k^n := \ensuremath{\Sigma}_n[{\rm e}^{-i k \ensuremath{\theta}} f(\ensuremath{\theta})]  = {1 \over n} \ensuremath{\sum}_{j=0}^{n-1} {\rm e}^{-i k \ensuremath{\theta}_j} f(\ensuremath{\theta}_j)
\]
\end{definition}

A remarkable fact is that the discete Fourier coefficients can be expressed as a sum of the true Fourier coefficients:

\begin{theorem}[discrete Fourier coefficients] If $\vchatf \ensuremath{\in} \ensuremath{\ell}^1$ (absolutely convergent Fourier coefficients) then
\[
\hat f_k^n = \ensuremath{\cdots} + \hat f_{k-2n} + \hat f_{k-n} + \hat f_k + \hat f_{k+n} + \hat f_{k+2n} + \ensuremath{\cdots}
\]
\end{theorem}
\textbf{Proof}
\begin{align*}
\hat f_k^n &= \ensuremath{\Sigma}_n[f(\ensuremath{\theta}) {\rm e}^{-{\rm i} k \ensuremath{\theta}}] = \ensuremath{\sum}_{\ensuremath{\ell}=-\ensuremath{\infty}}^\ensuremath{\infty} \hat f_\ensuremath{\ell} \ensuremath{\Sigma}_n[{\rm e}^{{\rm i} (\ensuremath{\ell}-k) \ensuremath{\theta}}] \\
&= \ensuremath{\sum}_{\ensuremath{\ell}=-\ensuremath{\infty}}^\ensuremath{\infty} \hat f_\ensuremath{\ell} \begin{cases} 1 & \ensuremath{\ell}-k = \ldots,-2n,-n,0,n,2n,\ldots  \cr
0 & \hbox{otherwise}
\end{cases}
\end{align*}
\ensuremath{\QED}

\begin{example}[Taylor coefficients via Geometric series] Consider the function
\[
f(\ensuremath{\theta}) = {2 \over 2 - {\rm e}^{{\rm i} \ensuremath{\theta}}}
\]
Under the change of variables $z = {\rm e}^{{\rm i} \ensuremath{\theta}}$ we know for $z$ on the unit circle this becomes (using the geometric series with $z/2$)
\[
{2 \over 2-z} = \ensuremath{\sum}_{k=0}^\ensuremath{\infty} {z^k \over 2^k}
\]
i.e., $\hat f_k = 1/2^k$ which is absolutely summable:
\[
\ensuremath{\sum}_{k=0}^\ensuremath{\infty} |\hat f_k| = f(0) = 2.
\]
If we use an $n$ point discretisation we get (using the geoemtric series with $2^{-n}$)
\[
\hat f_k^n = \hat f_k + \hat f_{k+n} + \hat f_{k+n} + \ensuremath{\cdots} = \ensuremath{\sum}_{p=0}^\ensuremath{\infty} {1 \over 2^{k+pn}} = {2^{n-k} \over 2^n - 1}
\]
Note that as $n \rightarrow \ensuremath{\infty}$, we have $\hat f_k^n \rightarrow \hat f_k$. \end{example}

Note that there is redundancy:

\begin{corollary}[aliasing] For all $p \ensuremath{\in} \ensuremath{\bbZ}$, $\hat f_k^n = \hat f_{k+pn}^n$.

\end{corollary}
\textbf{Proof} Follows immediately:
\[
\hat f_{k+pn}^n = \sum_{j=-\ensuremath{\infty}}^\ensuremath{\infty} \hat f_{k+(p+j)n}= \sum_{j=-\ensuremath{\infty}}^\ensuremath{\infty} \hat f_{k+j n} = \hat f_k^n.
\]
\ensuremath{\QED}

In other words if we know $\hat f_0^n, \ensuremath{\ldots}, \hat f_{n-1}^n$, we know $\hat f_k^n$ for all $k$ via a permutation, for example if $n = 2m+1$ we have
\[
\begin{bmatrix}
\hat f_{-m}^n \\
\ensuremath{\vdots}\\
\hat f_{-1}^n \\
\hat f_0^n \\
\ensuremath{\vdots}\\
\hat f_m^n
\end{bmatrix} = \underbrace{\begin{bmatrix} &&& 1 \\ &&&& \ensuremath{\ddots} \\ &&&&& 1 \\
    1 \\ & \ensuremath{\ddots} \\ && 1 \end{bmatrix}}_{P_\ensuremath{\sigma}}
\begin{bmatrix}
\hat f_0^n \\
\ensuremath{\vdots}\\
\hat f_m^n\\
\hat f_{m+1}^n \\
\ensuremath{\vdots}\\
\hat f_{n-1}^n
\end{bmatrix}
\]
where $\ensuremath{\sigma}$ has Cauchy notation (\emph{Careful}: we are using 1-based indexing here):
\[
\begin{pmatrix}
1 & 2 & \ensuremath{\cdots} & m & m+1 & m+2 & \ensuremath{\cdots} & n  \\
m+2 & m+3 & \ensuremath{\cdots} & n & 1 & 2 & \ensuremath{\cdots} & m+1
\end{pmatrix}.
\]
We can  prove \emph{convergence} whenever $f$ has absolutely summable coefficients. We will prove the result here in the special case where the negative coefficients are zero. That is, $\hat f_0^n, \ensuremath{\ldots}, \hat f_{n-1}^n$ are approximations of the Fourier\ensuremath{\endash}Taylor coefficients.

\begin{theorem}[Approximate Fourier-Taylor expansions converge] If $0 = \hat f_{-1} = \hat f_{-2} = \ensuremath{\cdots}$ and $\vchatf$ is absolutely convergent then
\[
f_n(\ensuremath{\theta}) = \ensuremath{\sum}_{k=0}^{n-1} \hat f_k^n {\rm e}^{{\rm i} k \ensuremath{\theta}}
\]
converges uniformly to $f(\ensuremath{\theta})$.

\end{theorem}
\textbf{Proof}
\begin{align*}
|f(\ensuremath{\theta}) - f_n(\ensuremath{\theta})| = |\ensuremath{\sum}_{k=0}^{n-1} (\hat f_k - \hat f_k^n) {\rm e}^{{\rm i} k \ensuremath{\theta}} + \ensuremath{\sum}_{k=n}^\ensuremath{\infty} \hat f_k {\rm e}^{{\rm i} k \ensuremath{\theta}}|
= |\ensuremath{\sum}_{k=n}^\ensuremath{\infty} \hat f_k ({\rm e}^{{\rm i} k \ensuremath{\theta}} - {\rm e}^{{\rm i} {\rm mod}(k,n) \ensuremath{\theta}})|
\ensuremath{\leq} 2 \ensuremath{\sum}_{k=n}^\ensuremath{\infty} |\hat f_k|
\end{align*}
which goes to zero as $n \ensuremath{\rightarrow} \ensuremath{\infty}$. \ensuremath{\QED}

For the general case we need to choose a range of coefficients that includes roughly an equal number of negative and positive coefficients (preferring negative over positive in a tie as a convention):
\[
f_n(\ensuremath{\theta}) = \ensuremath{\sum}_{k=-\ensuremath{\lceil}n/2\ensuremath{\rceil}}^{\ensuremath{\lfloor}n/2\ensuremath{\rfloor}} \hat f_k {\rm e}^{{\rm i} k \ensuremath{\theta}}
\]
In the problem sheet we will prove this converges provided the coefficients are absolutely convergent.





\section{Discrete Fourier Transform}
In the previous section we explored using the trapezium rule for approximating Fourier coefficients. This is a linear map from function values to coefficients and thus can be reinterpreted as a matrix-vector product, called the the Discrete Fourier Transform. It turns out the matrix is unitary which leads to important properties including interpolation. 

\textbf{Remark} A clever way of decomposing the DFT leads to a fast way of applying and inverting it, which is one of the most influencial algorithms of the 20th century: the Fast Fourier Transform. But this is beyond the scope of this module.

\subsection{The Discrete Fourier transform}
\begin{definition}[DFT] The \emph{Discrete Fourier Transform (DFT)} is defined as:
\begin{align*}
Q_n &:= {1 \over \sqrt{n}} \begin{bmatrix} 1 & 1 & 1&  \ensuremath{\cdots} & 1 \\
                                    1 & {\rm e}^{-\I \ensuremath{\theta}_1} & {\rm e}^{-\I \ensuremath{\theta}_2} & \ensuremath{\cdots} & {\rm e}^{-\I \ensuremath{\theta}_{n-1}} \\
                                    1 & {\rm e}^{-\I 2 \ensuremath{\theta}_1} & {\rm e}^{-\I 2 \ensuremath{\theta}_2} & \ensuremath{\cdots} & {\rm e}^{-\I 2\ensuremath{\theta}_{n-1}} \\
                                    \ensuremath{\vdots} & \ensuremath{\vdots} & \ensuremath{\vdots} & \ensuremath{\ddots} & \ensuremath{\vdots} \\
                                    1 & {\rm e}^{-\I (n-1) \ensuremath{\theta}_1} & {\rm e}^{-\I (n-1) \ensuremath{\theta}_2} & \ensuremath{\cdots} & {\rm e}^{-\I (n-1) \ensuremath{\theta}_{n-1}}
\end{bmatrix} \\
&= {1 \over \sqrt{n}} \begin{bmatrix} 1 & 1 & 1&  \ensuremath{\cdots} & 1 \\
                                    1 & \ensuremath{\omega}^{-1} & \ensuremath{\omega}^{-2} & \ensuremath{\cdots} & \ensuremath{\omega}^{-(n-1)}\\
                                    1 & \ensuremath{\omega}^{-2} & \ensuremath{\omega}^{-4} & \ensuremath{\cdots} & \ensuremath{\omega}^{-2(n-1)}\\
                                    \ensuremath{\vdots} & \ensuremath{\vdots} & \ensuremath{\vdots} & \ensuremath{\ddots} & \ensuremath{\vdots} \\
                                    1 & \ensuremath{\omega}^{-(n-1)} & \ensuremath{\omega}^{-2(n-1)} & \ensuremath{\cdots} & \ensuremath{\omega}^{-(n-1)^2}
\end{bmatrix}
\end{align*}
for the $n$-th root of unity $\ensuremath{\omega} = {\rm e}^{2\ensuremath{\pi}\I/n}$.  \end{definition}

Note that
\begin{align*}
Q_n^\ensuremath{\star} &= {1 \over \sqrt{n}} \begin{bmatrix}
1 & 1 & 1&  \ensuremath{\cdots} & 1 \\
1 & {\rm e}^{\I \ensuremath{\theta}_1} & {\rm e}^{\I 2 \ensuremath{\theta}_1} & \ensuremath{\cdots} & {\rm e}^{\I (n-1) \ensuremath{\theta}_1} \\
1 &  {\rm e}^{\I \ensuremath{\theta}_2}  & {\rm e}^{\I 2 \ensuremath{\theta}_2} & \ensuremath{\cdots} & {\rm e}^{\I (n-1)\ensuremath{\theta}_2} \\
\ensuremath{\vdots} & \ensuremath{\vdots} & \ensuremath{\vdots} & \ensuremath{\ddots} & \ensuremath{\vdots} \\
1 & {\rm e}^{\I \ensuremath{\theta}_{n-1}} & {\rm e}^{\I 2 \ensuremath{\theta}_{n-1}} & \ensuremath{\cdots} & {\rm e}^{\I (n-1) \ensuremath{\theta}_{n-1}}
\end{bmatrix} \\
&= {1 \over \sqrt{n}} \begin{bmatrix}
1 & 1 & 1&  \ensuremath{\cdots} & 1 \\
1 & \ensuremath{\omega}^{1} & \ensuremath{\omega}^{2} & \ensuremath{\cdots} & \ensuremath{\omega}^{(n-1)}\\
1 & \ensuremath{\omega}^{2} & \ensuremath{\omega}^{4} & \ensuremath{\cdots} & \ensuremath{\omega}^{2(n-1)}\\
\ensuremath{\vdots} & \ensuremath{\vdots} & \ensuremath{\vdots} & \ensuremath{\ddots} & \ensuremath{\vdots} \\
1 & \ensuremath{\omega}^{(n-1)} & \ensuremath{\omega}^{2(n-1)} & \ensuremath{\cdots} & \ensuremath{\omega}^{(n-1)^2}
\end{bmatrix}
\end{align*}
Hence we have
\[
\underbrace{\begin{bmatrix} \hat f_0^n \\ \ensuremath{\vdots} \\ \hat f_{n-1}^n \end{bmatrix}}_{\vchatf^n} =
{1 \over \sqrt{n}} Q_n \underbrace{\begin{bmatrix} f(\ensuremath{\theta}_0) \\ \ensuremath{\vdots} \\ f(\ensuremath{\theta}_{n-1}) \end{bmatrix}}_{\ensuremath{\bm{\f}}^n}
\]
The choice of normalisation constant is motivated by the following:

\textbf{Proposition 1 (DFT is Unitary)} $Q_n \ensuremath{\in} U(n)$, that is, $Q_n^\ensuremath{\star} Q_n = Q_n Q_n^\ensuremath{\star} = I$.

\textbf{Proof}
\[
Q_n Q_n^\ensuremath{\star}  = \begin{bmatrix} \ensuremath{\Sigma}_n[1] & \ensuremath{\Sigma}_n[{\rm e}^{\I \ensuremath{\theta}}] & \ensuremath{\cdots} & \ensuremath{\Sigma}_n[{\rm e}^{\I (n-1) \ensuremath{\theta}}] \\
                            \ensuremath{\Sigma}_n[{\rm e}^{-\I \ensuremath{\theta}}] & \ensuremath{\Sigma}_n[1] & \ensuremath{\cdots} & \ensuremath{\Sigma}_n[{\rm e}^{\I (n-2) \ensuremath{\theta}}] \\
                            \ensuremath{\vdots} & \ensuremath{\vdots} & \ensuremath{\ddots} & \ensuremath{\vdots} \\
                            \ensuremath{\Sigma}_n[{\rm e}^{-\I(n-1) \ensuremath{\theta}}] & \ensuremath{\Sigma}_n[{\rm e}^{-\I(n-2) \ensuremath{\theta}}] & \ensuremath{\cdots} & \ensuremath{\Sigma}_n[1]
                            \end{bmatrix} = I
\]
\ensuremath{\QED}

In other words, $Q_n$ is easily inverted and we also have a map from discrete Fourier coefficients back to values:
\[
\sqrt{n} Q_n^\ensuremath{\star} \vchatf^n = \ensuremath{\bm{\f}}^n
\]
\begin{example}[Computing Sum] Define the following infinite sum (which has no name apparently, according to Mathematica):
\[
S_n(k) := \ensuremath{\sum}_{p=0}^\ensuremath{\infty} {1 \over (k+pn)!}
\]
We can use the DFT to compute $S_n(0), \ensuremath{\ldots}, S_n(n-1)$. Consider
\[
f(\ensuremath{\theta}) = \exp({\rm e}^{\I \ensuremath{\theta}}) = \ensuremath{\sum}_{k=0}^\ensuremath{\infty} {{\rm e}^{\I k \ensuremath{\theta}} \over k!}
\]
where we know the Fourier coefficients from the Taylor series of ${\rm e}^z$. The discrete Fourier coefficients satisfy for $0 \ensuremath{\leq} k \ensuremath{\leq} n-1$:
\[
\hat f_k^n = \hat f_k + \hat f_{k+n} + \hat f_{k+2n} + \ensuremath{\cdots} = S_n(k)
\]
Thus we have
\[
\begin{bmatrix}
S_n(0) \\
\ensuremath{\vdots} \\
S_n(n-1)
\end{bmatrix} = {1 \over \sqrt{n}} Q_n \begin{bmatrix} 1 \\
                                \exp({\rm e}^{2\I \ensuremath{\pi}/n}) \\
                                \ensuremath{\vdots} \\
                                \exp({\rm e}^{2\I (n-1) \ensuremath{\pi}/n}) \end{bmatrix}
\]
\end{example}

\subsection{Interpolation}
We investigated  interpolation and least squares using polynomials at evenly spaced points, observing that there were issues with stability. We now show that the DFT actually gives coefficients that interpolate using Fourier expansions. As the DFT is a unitary matrix multiplication is \ensuremath{\ldq}stable", i.e. it preserves norms and hence we know it cannot cause the same huge blow-up we saw for polynomials. That is: whilst polynomials are bad for interpolation at evenly spaced points, trigonometric polynomials are great. 

The following guarantees that our approximate Fourier series actually interpolates the data:

\begin{corollary}[Interpolation]
\[
f_n(\ensuremath{\theta}) := \ensuremath{\sum}_{k=0}^{n-1} \hat f_k^n {\rm e}^{\I k \ensuremath{\theta}}
\]
interpolates $f$ at $\ensuremath{\theta}_j$:
\[
f_n(\ensuremath{\theta}_j) = f(\ensuremath{\theta}_j)
\]
\end{corollary}
\textbf{Proof} We have
\[
f_n(\ensuremath{\theta}_j) = \ensuremath{\sum}_{k=0}^{n-1} \hat f_k^n {\rm e}^{\I k \ensuremath{\theta}_j} = \sqrt n \ensuremath{\bm{\e}}_j^\ensuremath{\top} Q_n^\ensuremath{\star} \vchatf^n = \ensuremath{\bm{\e}}_j^\ensuremath{\top} Q_n^\ensuremath{\star} Q_n \ensuremath{\bm{\f}}^n = f(\ensuremath{\theta}_j).
\]
\ensuremath{\QED}

\begin{example}[DFT versus Lagrange] Consider interpolating $f(z) = \exp z$ by a polynomial at the points $1, \I, -1, -\I$. We can use Lagrange polynomials:
\meeq{
\ensuremath{\ell}_1(z) ={ (z - \I)(z + 1)(z + \I) \over 2(1 - \I)(1 + \I)} = { z^3 + z^2 + z + 1 \over 4} \ccr
\ensuremath{\ell}_2(z) ={ (z - 1)(z + 1)(z + \I) \over (\I - 1) (\I + 1) 2\I} = { \I z^3 - z^2 - \I z + 1 \over 4} \ccr
\ensuremath{\ell}_3(z) ={ (z - 1)(z - \I)(z + \I) \over -2 (-1-\I)(-1+\I)} = {-z^3 + z^2 - z + 1 \over 4} \ccr
\ensuremath{\ell}_4(z) ={ (z - 1)(z - \I)(z+1) \over (-\I-1)(-2\I)(-\I+1)} = {- \I z^3 -z^2 + \I z + 1 \over 4}
}
So we get the interpolant:
\begin{align*}
\E & \ensuremath{\ell}_1(z) + \E^\I \ensuremath{\ell}_2(z) + \E^{-1} \ensuremath{\ell}_3(z) + \E^{-\I} \ensuremath{\ell}_4(z) \\
 &= 
{\E + \E^\I + \E^{-1} + \E^{-\I} \over 4} +
{\E - \I \E^\I - \E^{-1} + \I \E^{-\I} \over 4} z +
 {\E - \E^\I + \E^{-1} - \I \E^{-\I} \over 4} z^2 +
 {\E + \I \E^\I - \E^{-1} - \I \E^{-\I} \over 4} z^3 
\end{align*}
Alternatively we could have deduced this directly from the DFT. In particular, we know the coefficients of the interpolating polynomial must be, for $\ensuremath{\omega} = \I$,
\[
\Vectt[\hat f_0^4, \hat f_1^4, \hat f_2^4, \hat f_3^4] = 
{1 \over 4} \begin{bmatrix}1 & 1 & 1 & 1 \\
                            1 & -\I & -1 & \I \\
                            1 & -1 & 1 & -1 \\
                            1 & \I & -1 & -\I
                            \end{bmatrix}
 \Vectt[\E, \E^\I, \E^{-1}, \E^{-\I}] = {1 \over 4} \Vectt[\E + \E^\I + \E^{-1} + \E^{-\I} ,
 \E -\I \E^\I - \E^{-1} + \I \E^{-\I} ,
 \E - \E^\I + \E^{-1} - \E^{-\I} ,
 \E + \I \E^\I - \E^{-1} = \I \E^{-\I}
 ]
\]
\end{example}

The interpolation property also applies to the approximation
\[
f_n(\ensuremath{\theta}) = \ensuremath{\sum}_{k=-\ensuremath{\lceil}n/2\ensuremath{\rceil}}^{\ensuremath{\lfloor}n/2\ensuremath{\rfloor}} \hat f_k {\rm e}^{{\rm i} k \ensuremath{\theta}}
\]
for general Fourier series, which is investigated in the problem sheet.






\appendix

\chapter{Asymptotics and Computational Cost}
\input{A.Asymptotics}

\chapter{Permutation Matrices}

Permutation matrices are matrices that represent the action of permuting the entries of a vector, that is, matrix representations of the symmetric group $S_n$, acting on $\ensuremath{\bbR}^n$. Recall every $\ensuremath{\sigma} \ensuremath{\in} S_n$ is a bijection between $\{1,2,\ensuremath{\ldots},n\}$ and itself. We can write a permutation $\ensuremath{\sigma}$ in \emph{Cauchy notation}:
\[
\begin{pmatrix}
 1 & 2 & 3 & \ensuremath{\cdots} & n \cr
 \ensuremath{\sigma}_1 & \ensuremath{\sigma}_2 & \ensuremath{\sigma}_3 & \ensuremath{\cdots} & \ensuremath{\sigma}_n
 \end{pmatrix}
\]
where $\{\ensuremath{\sigma}_1,\ensuremath{\ldots},\ensuremath{\sigma}_n\} = \{1,2,\ensuremath{\ldots},n\}$ (that is, each integer appears precisely once). We denote the \emph{inverse permutation} by $\ensuremath{\sigma}^{-1}$, which can be constructed by swapping the rows of the Cauchy notation and reordering.

We can encode a permutation in vector $\mathbf \ensuremath{\sigma} = [\ensuremath{\sigma}_1,\ensuremath{\ldots},\ensuremath{\sigma}_n]$.  This induces an action on a vector (using indexing notation)
\[
\ensuremath{\bm{\v}}[\mathbf \ensuremath{\sigma}] = \begin{bmatrix}v_{\ensuremath{\sigma}_1}\\ \vdots \\ v_{\ensuremath{\sigma}_n} \end{bmatrix}
\]
\begin{example}[permutation of a vector]  Consider the permutation $\ensuremath{\sigma}$ given by
\[
\begin{pmatrix}
 1 & 2 & 3 & 4 & 5 \cr
 1 & 4 & 2 & 5 & 3
 \end{pmatrix}
\]
We can apply it to a vector:


\begin{lstlisting}
(*@\HLJLk{using}@*) (*@\HLJLn{LinearAlgebra}@*)
(*@\HLJLn{\ensuremath{\sigma}}@*) (*@\HLJLoB{=}@*) (*@\HLJLp{[}@*)(*@\HLJLni{1}@*)(*@\HLJLp{,}@*) (*@\HLJLni{4}@*)(*@\HLJLp{,}@*) (*@\HLJLni{2}@*)(*@\HLJLp{,}@*) (*@\HLJLni{5}@*)(*@\HLJLp{,}@*) (*@\HLJLni{3}@*)(*@\HLJLp{]}@*)
(*@\HLJLn{v}@*) (*@\HLJLoB{=}@*) (*@\HLJLp{[}@*)(*@\HLJLni{6}@*)(*@\HLJLp{,}@*) (*@\HLJLni{7}@*)(*@\HLJLp{,}@*) (*@\HLJLni{8}@*)(*@\HLJLp{,}@*) (*@\HLJLni{9}@*)(*@\HLJLp{,}@*) (*@\HLJLni{10}@*)(*@\HLJLp{]}@*)
(*@\HLJLn{v}@*)(*@\HLJLp{[}@*)(*@\HLJLn{\ensuremath{\sigma}}@*)(*@\HLJLp{]}@*) (*@\HLJLcs{{\#}}@*) (*@\HLJLcs{we}@*) (*@\HLJLcs{permutate}@*) (*@\HLJLcs{entries}@*) (*@\HLJLcs{of}@*) (*@\HLJLcs{v}@*)
\end{lstlisting}

\begin{lstlisting}
5-element Vector(*@{{\{}}@*)Int64(*@{{\}}}@*):
  6
  9
  7
 10
  8
\end{lstlisting}


Its inverse permutation $\ensuremath{\sigma}^{-1}$ has Cauchy notation coming from swapping the rows of the Cauchy notation of $\ensuremath{\sigma}$ and sorting:
\[
\begin{pmatrix}
 1 & 4 & 2 & 5 & 3 \cr
 1 & 2 & 3 & 4 & 5
 \end{pmatrix} \rightarrow \begin{pmatrix}
 1 & 2 & 4 & 3 & 5 \cr
 1 & 3 & 2 & 5 & 4
 \end{pmatrix} 
\]
\end{example}

Note that the operator
\[
P_\ensuremath{\sigma}(\ensuremath{\bm{\v}}) = \ensuremath{\bm{\v}}[{\mathbf \ensuremath{\sigma}}]
\]
is linear in $\ensuremath{\bm{\v}}$, therefore, we can identify it with a matrix whose action is:
\[
P_\ensuremath{\sigma} \begin{bmatrix} v_1\\ \vdots \\ v_n \end{bmatrix} = \begin{bmatrix}v_{\ensuremath{\sigma}_1} \\ \vdots \\ v_{\ensuremath{\sigma}_n}  \end{bmatrix}.
\]
The entries of this matrix are
\[
P_\ensuremath{\sigma}[k,j] = \ensuremath{\bm{\e}}_k^\ensuremath{\top} P_\ensuremath{\sigma} \ensuremath{\bm{\e}}_j = \ensuremath{\bm{\e}}_k^\ensuremath{\top} \ensuremath{\bm{\e}}_{\ensuremath{\sigma}^{-1}_j} = \ensuremath{\delta}_{k,\ensuremath{\sigma}^{-1}_j} = \ensuremath{\delta}_{\ensuremath{\sigma}_k,j}
\]
where $\ensuremath{\delta}_{k,j}$ is the \emph{Kronecker delta}:
\[
\ensuremath{\delta}_{k,j} := \begin{cases} 1 & k = j \\
                        0 & \hbox{otherwise}
                        \end{cases}.
\]
This construction motivates the following definition:

\begin{definition}[permutation matrix] $P \in \ensuremath{\bbR}^{n \ensuremath{\times} n}$ is a permutation matrix if it is equal to the identity matrix with its rows permuted. \end{definition}

\begin{proposition}[permutation matrix inverse]  Let $P_\ensuremath{\sigma}$ be a permutation matrix corresponding to the permutation $\ensuremath{\sigma}$. Then
\[
P_\ensuremath{\sigma}^\ensuremath{\top} = P_{\ensuremath{\sigma}^{-1}} = P_\ensuremath{\sigma}^{-1}
\]
That is, $P_\ensuremath{\sigma}$ is \emph{orthogonal}:
\[
P_\ensuremath{\sigma}^\ensuremath{\top} P_\ensuremath{\sigma} = P_\ensuremath{\sigma} P_\ensuremath{\sigma}^\ensuremath{\top} = I.
\]
\end{proposition}
\textbf{Proof}

We prove orthogonality via:
\[
\ensuremath{\bm{\e}}_k^\ensuremath{\top} P_\ensuremath{\sigma}^\ensuremath{\top} P_\ensuremath{\sigma} \ensuremath{\bm{\e}}_j = (P_\ensuremath{\sigma} \ensuremath{\bm{\e}}_k)^\ensuremath{\top} P_\ensuremath{\sigma} \ensuremath{\bm{\e}}_j = \ensuremath{\bm{\e}}_{\ensuremath{\sigma}^{-1}_k}^\ensuremath{\top} \ensuremath{\bm{\e}}_{\ensuremath{\sigma}^{-1}_j} = \ensuremath{\delta}_{k,j}
\]
This shows $P_\ensuremath{\sigma}^\ensuremath{\top} P_\ensuremath{\sigma} = I$ and hence $P_\ensuremath{\sigma}^{-1} = P_\ensuremath{\sigma}^\ensuremath{\top}$. 

\ensuremath{\QED}







\end{document}